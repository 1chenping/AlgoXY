\ifx\wholebook\relax \else
% ------------------------

\documentclass[UTF8]{article}
%------------------- Other types of document example ------------------------
%
%\documentclass[twocolumn]{IEEEtran-new}
%\documentclass[12pt,twoside,draft]{IEEEtran}
%\documentstyle[9pt,twocolumn,technote,twoside]{IEEEtran}
%
%-----------------------------------------------------------------------------
%
% loading packages
%

\RequirePackage{ifpdf}
\RequirePackage{ifxetex}

%
%
\ifpdf
  \RequirePackage[pdftex,%
       bookmarksnumbered,%
              colorlinks,%
          linkcolor=blue,%
              hyperindex,%
        plainpages=false,%
       pdfstartview=FitH]{hyperref}
\else\ifxetex
  \RequirePackage[bookmarksnumbered,%
               colorlinks,%
           linkcolor=blue,%
               hyperindex,%
         plainpages=false,%
        pdfstartview=FitH]{hyperref}
\else
  \RequirePackage[dvipdfm,%
        bookmarksnumbered,%
               colorlinks,%
           linkcolor=blue,%
               hyperindex,%
         plainpages=false,%
        pdfstartview=FitH]{hyperref}
\fi\fi
%\usepackage{hyperref}

% other packages
%--------------------------------------------------------------------------
\usepackage{graphicx, color}
\usepackage{subfig}
\usepackage{tikz}
\usetikzlibrary{matrix,positioning}

\usepackage{amsmath, amsthm, amssymb} % for math
\usepackage{exercise} % for exercise
\usepackage{import} % for nested input

%
% for programming
%
\usepackage{verbatim}
\usepackage{listings}
%\usepackage{algorithmic} %old version; we can use algorithmicx instead
\usepackage{algorithm}
\usepackage[noend]{algpseudocode} %for pseudo code, include algorithmicsx automatically
\usepackage{appendix}
\usepackage{makeidx} % for index support
\usepackage{titlesec}

\usepackage[cm-default]{fontspec}
\usepackage{xunicode}

% detect and select Chinese font
% ------------------------------
% the following cmd can list all availabe Chinese fonts in host.
% fc-list :lang=zh
\def\myfont{STHeiti}  % Under Mac OS X
\def\linuxfallback{WenQuanYi Micro Hei} % Under Linux
\def\winfallback{SimSun} % Under Windows
\suppressfontnotfounderror1 % Avoid setting exit code (error level) to break make process
\count255=\interactionmode
\batchmode
\font\foo="\myfont"\space at 10pt
\ifx\foo\nullfont
  \font\foo = "\linuxfallback"\space at 10pt
  \ifx\foo\nullfont
    \font\foo = "\winfallback"\space at 10pt
    \ifx\foo\nullfont
      \errorstopmode
      \errmessage{no suitable Chinese font found}
    \else
      \let\myfont=\winfallback % Windows
    \fi
  \else
    \let\myfont=\linuxfallback % Linux
  \fi
\fi
\interactionmode=\count255
\setmainfont[Mapping=tex-text]{\myfont}

\XeTeXlinebreaklocale "zh"  % to solve the line breaking issue
\XeTeXlinebreakskip = 0pt plus 1pt minus 0.1pt

\titleformat{\paragraph}
{\normalfont\normalsize\bfseries}{\theparagraph}{1em}{}
\titlespacing*{\paragraph}
{0pt}{3.25ex plus 1ex minus .2ex}{1.5ex plus .2ex}

\lstdefinelanguage{Smalltalk}{
  morekeywords={self,super,true,false,nil,thisContext}, % This is overkill
  morestring=[d]',
  morecomment=[s]{"}{"},
  alsoletter={\#:},
  escapechar={!},
  literate=
    {BANG}{!}1
    {UNDERSCORE}{\_}1
    {\\st}{Smalltalk}9 % convenience -- in case \st occurs in code
    % {'}{{\textquotesingle}}1 % replaced by upquote=true in \lstset
    {_}{{$\leftarrow$}}1
    {>>>}{{\sep}}1
    {^}{{$\uparrow$}}1
    {~}{{$\sim$}}1
    {-}{{\sf -\hspace{-0.13em}-}}1  % the goal is to make - the same width as +
    %{+}{\raisebox{0.08ex}{+}}1		% and to raise + off the baseline to match -
    {-->}{{\quad$\longrightarrow$\quad}}3
	, % Don't forget the comma at the end!
  tabsize=2
}[keywords,comments,strings]

% for better Haskell code outlook
\lstdefinelanguage{Haskell}{
  basicstyle=\small\ttfamily,
  flexiblecolumns=false,
  basewidth={0.5em,0.45em},
  literate={+}{{$+$}}1 {/}{{$/$}}1 {*}{{$*$}}1 {=}{{$=$}}1
           {>}{{$>$}}1 {<}{{$<$}}1 {\\}{{$\lambda$}}1
           {\\\\}{{\char`\\\char`\\}}1
           {->}{{$\rightarrow$}}2 {>=}{{$\geq$}}2 {<-}{{$\leftarrow$}}2
           {<=}{{$\leq$}}2 {=>}{{$\Rightarrow$}}2
           {\ .}{{$\circ$}}2 {\ .\ }{{$\circ$}}2
           {>>}{{>>}}2 {>>=}{{>>=}}2
           {|}{{$\mid$}}1
}[keywords,comments,strings]

\lstloadlanguages{C, C++, Lisp, Haskell, Python, Smalltalk}

\lstset{
  showstringspaces = false
}

% ======================================================================

\def\BibTeX{{\rm B\kern-.05em{\sc i\kern-.025em b}\kern-.08em
    T\kern-.1667em\lower.7ex\hbox{E}\kern-.125emX}}

%
% mathematics
%
\newcommand{\be}{\begin{equation}}
\newcommand{\ee}{\end{equation}}
\newcommand{\bmat}[1]{\left( \begin{array}{#1} }
\newcommand{\emat}{\end{array} \right) }
\newcommand{\VEC}[1]{\mbox{\boldmath $#1$}}

% numbered equation array
\newcommand{\bea}{\begin{eqnarray}}
\newcommand{\eea}{\end{eqnarray}}

% equation array not numbered
\newcommand{\bean}{\begin{eqnarray*}}
\newcommand{\eean}{\end{eqnarray*}}

\newtheorem{theorem}{Theorem}[section]
\newtheorem{lemma}[theorem]{Lemma}
\newtheorem{proposition}[theorem]{Proposition}
\newtheorem{corollary}[theorem]{Corollary}


\setcounter{page}{1}

\begin{document}

%--------------------------

% ================================================================
%                 COVER PAGE
% ================================================================

\title{二叉堆}

\author{刘新宇
\thanks{{\bfseries 刘新宇 } \newline
  Email: liuxinyu95@gmail.com \newline}
  }

\maketitle
\fi

\markboth{二叉堆}{初等算法}

\ifx\wholebook\relax
\chapter{二叉堆}
\numberwithin{Exercise}{chapter}
\fi

% ================================================================
%                 Introduction
% ================================================================
\section{简介}
\label{introduction}
\index{二叉堆}

堆是被广泛应用的一种数据结构。堆可以用于解决很多实际问题,包括排序、带有优先级的调度,实现图算法等等\cite{wiki-heap}。

堆有很多不同的实现,其中最常见的一种通过数组来表示二叉树\cite{CLRS},进而实现堆。例如C++标准库STL中的heap,和Python库中的heapq都是这样实现堆的。由R.W. Floyd给出的最高效的堆排序算法也是利用这个实现\cite{wiki-heapsort}\cite{rosetta-heapsort}。

堆是一种通用的概念,它也可以用数组以外的其他数据结构来实现。本章中,我们给出一些使用二叉树来实现的堆,包括左偏堆(Leftist Heap)、Skew堆、和Splay堆。它们非常适合纯函数式实现i\cite{okasaki-book}。

堆是一种满足如下性质的数据结构:
\begin{itemize}
\item 顶部(top)总是保存着最小(或最大)的元素;
\item 弹出(pop)操作将顶部元素移除,同时保持堆的性质,新的顶部元素仍然是剩余元素中的最小(或最大)值;
\item 将新元素插入到堆中仍然保持堆的性质,顶部元素还是所有元素中的最大(或最小)值;
\item 其他操作(例如将两个堆合并),都会保持堆的性质。
\end{itemize}

这一定义是递归的。它并没有限定实现堆的低层数据结构。

我们称顶部保存最小元素的堆为\underline{最小堆},顶部保存最大元素的堆为\underline{最大堆}。

% ================================================================
%                 Implicit binary heap by array
% ================================================================
\section{用数组实现隐式二叉堆}
\label{ibheap}
\index{隐式二叉堆}

考虑堆的定义,我们可以用树来实现堆。一种直观的想法是将最小(或最大)元素保存在树的根节点。获取“顶部”元素时,我们可以直接返回根节点中的数据。执行“弹出”操作时,我们将根节点删除,然后从子节点中重建树。

我们称使用二叉树实现的堆为\underline{二叉堆}。本章介绍三种不同的二叉堆实现。

% ================================================================
%                 Definition
% ================================================================
\subsection{定义}

第一种实现称为隐式二叉树。考虑如何用数组来表示一棵完全二叉树(例如,有些编程语言中没有结构或记录等复合数据类型,只能使用数组来定义二叉树)。我们可以将全部元素自顶向下(从根节点开始,到叶子节点为止)压缩放入数组中。

图\ref{fig:tree-array-map}展示了一棵完全二叉树和它相应的数组表示形式。

\begin{figure}[htbp]
       \begin{center}
       	  \includegraphics[scale=0.5]{img/tree-array-map-tree.ps}
          \includegraphics[scale=0.5]{img/tree-array-map-array.ps}
        \caption{完全二叉树到数组的映射。} \label{fig:tree-array-map}
       \end{center}
\end{figure}

这一树和数组之间的映射可以定义如下(令数组的索引从1开始):

\begin{algorithmic}[1]
\Function{Parent}{$i$}
  \State \Return $\lfloor \frac{i}{2} \rfloor$
\EndFunction
\Statex
\Function{Left}{$i$}
  \State \Return $2i$
\EndFunction
\Statex
\Function{Right}{$i$}
  \State \Return $2i+1$
\EndFunction
\end{algorithmic}

对数组中任意第$i$个元素代表的节点,由于二叉树是完全的,我们可以通过定位到第$\lfloor i/2 \rfloor$个元素找到它的父节点;它的左子树对应第$2i$个元素,而右子树对应第$2i+1$个元素。如果子节点的索引超出了数组的长度,说明这它不含有相应的子树(例如叶子节点)。

在实际的应用中,父节点和子树的访问可以通过位运算实现,例如下面的C代码。注意,代码中的索引从0开始。

\lstset{language=C}
\begin{lstlisting}
#define PARENT(i) ((((i) + 1) >> 1) - 1)

#define LEFT(i) (((i) << 1) + 1)

#define RIGHT(i) (((i) + 1) << 1)
\end{lstlisting}

% ================================================================
%                 Heapify
% ================================================================
\subsection{Heapify}
\index{二叉堆!Heapify}

堆算法中最重要的部份就是维护堆的性质:即顶部元素为最小(或最大)元素。

对于用数组表示的二叉堆,给定任何索引为$i$的节点,我们可以检查它的两个子节点是否都不小于父节点的元素。如果不满足,我们可以通过递归的交换,使得父节点保存最小值\cite{CLRS}。注意:这里我们假设$i$的两棵子树都是合法的堆。

下面的算法从给定的数组索引开始,迭代检查所有的子节点以保持最小堆性质。

\begin{algorithmic}[1]
\Function{Heapify}{$A, i$}
  \State $n \gets |A|$
  \Loop
    \State $l \gets$ \Call{Left}{$i$}
    \State $r \gets$ \Call{Right}{$i$}
    \State $smallest \gets i$
    \If{$l < n \land A[l] < A[i]$}
      \State $smallest \gets l$
    \EndIf
    \If{$r < n \land A[r] < A[smallest]$}
      \State $smallest \gets r$
    \EndIf
    \If{$smallest \neq i$}
      \State \textproc{Exchange} $A[i] \leftrightarrow A[smallest]$
      \State $i \gets smallest$
    \Else
      \State \Return
    \EndIf
  \EndLoop
\EndFunction
\end{algorithmic}

算法接受一个数组$A$和一个索引$i$,$A[i]$的两个子节点都不应比它小。否则,我选出最小的元素保存在$A[i]$,并将较大的元素交换至子树,然后算法自顶向下检查并修复堆的性质直到叶子节点或者没有发现任何违反堆性质的情况。

\textproc{Heapify}的时间复杂度为$O(\lg n)$,其中$n$是元素的总数。这是因为上述算法中的循环次数和完全二叉树的高度成正比。

在具体的实现中,元素之间的比较运算可以用参数的形式传入,这样同一实现就即可以支持最小堆,也支持最大堆。下面的C例子程序实现了这一算法。

\begin{lstlisting}
typedef int (*Less)(Key, Key);
int less(Key x, Key y) { return x < y; }
int notless(Key x, Key y) { return !less(x, y); }

void heapify(Key* a, int i, int n, Less lt) {
    int l, r, m;
    while (1) {
        l = LEFT(i);
        r = RIGHT(i);
        m = i;
        if (l < n && lt(a[l], a[i]))
            m = l;
        if (r < n && lt(a[r], a[m]))
            m = r;
        if (m != i) {
            swap(a, i, m);
            i = m;
        } else
            break;
    }
}
\end{lstlisting}

图\ref{fig:heapify}描述了\textproc{Heapify}从索引2开始,以最大堆处理数组$\{16, 4, 10, 14, 7, 9, 3, 2, 8, 1\}$过程中的各个步骤。数组最终变换为$\{16, 14, 10, 8, 7, 9, 3, 2, 4, 1\}$。

\begin{figure}[htbp]
    \centering
    \subfloat[步骤1:4、14和7中的最大元素是14。将4和左侧子节点交换;]{\includegraphics[scale=0.5]{img/heapify-1.ps}} \hspace{0.01\textwidth}
    \subfloat[步骤2:2、4和8中的最大元素是8。将4和右侧子节点交换;]{\includegraphics[scale=0.5]{img/heapify-2.ps}} \\
    \subfloat[4为叶子节点。过程结束。]{\includegraphics[scale=0.5]{img/heapify-3.ps}}
    \caption{Heapify的例子,堆为最大堆} \label{fig:heapify}
\end{figure}


% ================================================================
%                 Build a heap
% ================================================================
\subsection{构造堆}
\index{二叉堆!构造堆}

使用\textproc{Heapify}算法,我们可以很方便地从任意数组构造堆。观察完全二叉树各层的节点数:

$1, 2, 4, 8, ..., 2^i, ...$.

唯一例外是最后一层,由于树并不一定是满的(完全二叉树不等同于满),最后一层最多含有$2^{p-1}$个节点,其中$2^p \leq n$,$n$是数组的长度。

\textproc{Heapify}算法对于叶子节点不起任何作用,这是由于所有的叶子节点都已经满足堆性质了。我们可以跳过所有的叶子节点,从第一个分支节点开始执行\textproc{Heapify}。显然第一个分支节点的索引不大于$\lfloor n/2 \rfloor$。

根据这一分析,我们可以设计出如下的堆构造算法(以最小堆为例):

\begin{algorithmic}[1]
\Function{Build-Heap}{$A$}
  \State $n \gets |A|$
  \For{$i \gets \lfloor n/2 \rfloor$ down to $1$}
    \State \Call{Heapify}{$A, i$}
  \EndFor
\EndFunction
\end{algorithmic}

虽然\textproc{Heapify}算法的复杂度为$O(\lg n)$,但是\textproc{Build-Heap}的复杂度不是$O(n \lg n)$,而是线性时间$O(n)$的。我们跳过了所有的叶子节点,最多有$1/4$一半的节点被比较并向下移动一次;最多有$1/8$的节点被比较并向下移动两次;最多有$1/16$的节点被比较并向下移动三次……总共比较和移动次数的上限为:

\be
S = n (\frac{1}{4} + 2 \frac{1}{8} + 3 \frac{1}{16} + ...)
\label{eq:build-heap-1}
\ee

将两侧都乘以2:

\be
2S = n (\frac{1}{2} + 2 \frac{1}{4} + 3 \frac{1}{8} + ...)
\label{eq:build-heap-2}
\ee

用式(\ref{eq:build-heap-2})减去式(\ref{eq:build-heap-1}),我们有:

\[
S = n (\frac{1}{2} + \frac{1}{4} + \frac{1}{8} + ...) = n
\]

下面的C语言例子程序实现了堆构造算法:

\lstset{language=C}
\begin{lstlisting}
void build_heap(Key* a, int n, Less lt) {
    int i;
    for (i = (n-1) >> 1; i >= 0; --i)
        heapify(a, i, n, lt);
}
\end{lstlisting}

图\ref{fig:build-heap-1}、\ref{fig:build-heap-2}和\ref{fig:build-heap-3}描述了从数组$\{4, 1, 3, 2, 16, 9, 10, 14, 8, 7\}$构造一个最大堆的各个步骤。黑色节点表示执行\textproc{Heapify}时开始的节点;灰色节点表示为了维持堆性质进行交换的节点。

\begin{figure}[htbp]
    \centering
    \subfloat[开始构造前的一个无序数组。]{\includegraphics[scale=0.5]{img/build-heap-array.ps}} \\
    \subfloat[第一步:数组被映射为二叉树。第一个分支节点的值是16。]{\includegraphics[scale=0.5]{img/build-heap-1.ps}} \\
    \subfloat[第二步:16是当前子树中的最大元素,接下来检查元素2所在的节点。]{\includegraphics[scale=0.5]{img/build-heap-2.ps}}
    \caption{从任意数组构造堆。灰色表示每步中进行交换的节点,黑色表示下一步需要检查的节点。} \label{fig:build-heap-1}
\end{figure}

\begin{figure}[htbp]
    \centering
    \subfloat[第三步:14是子树中的最大元素,交换2和14;接下来检查元素为3的节点。]{\includegraphics[scale=0.5]{img/build-heap-3.ps}} \\
    \subfloat[第四步:10是子树中的最大元素,交换10和3;接下来检查元素为1的节点。]{\includegraphics[scale=0.5]{img/build-heap-4.ps}}
    \caption{从任意数组构造堆。灰色表示每步中进行交换的节点,黑色表示下一步需要检查的节点。} \label{fig:build-heap-2}
\end{figure}

\begin{figure}[htbp]
    \centering
    \subfloat[第五步:16是子树中的最大元素,先交换16和1,接下来交换1和7;此后检查元素为4的根节点。]{\includegraphics[scale=0.5]{img/build-heap-5.ps}} \\
    \subfloat[第六步:交换4和14,然后交换4和8;堆构造结束。]{\includegraphics[scale=0.5]{img/build-heap-6.ps}}
    \caption{从任意数组构造堆。灰色表示每步中进行交换的节点,黑色表示下一步需要检查的节点。} \label{fig:build-heap-3}
\end{figure}

% ================================================================
%                 Basic heap operations
% ================================================================
\subsection{堆的基本操作}

堆的通用定义要求我们提供一些基本操作使得用户可以获取或者改变数据。

最重要的操作包括获取顶部元素(查找最小或最大元素),弹出顶部元素,寻找最小(或最大)的前$k$个元素,减小某一元素的值(此操作对应最小堆,最大堆的相应操作是增加某一元素的值),以及插入新元素。

对于用完全二叉树实现的堆,大部份操作的复杂度在最差情况下都是$O(\lg n)$的。有些操作,例如获取顶部元素,仅仅需要常数时间($O(1)$)。

\subsubsection{获取顶部元素}
\index{二叉堆!top}

在用二叉树实现的堆中,根节点保存了最小(或最大)元素。根节点对应数组的第一个值。

\begin{algorithmic}[1]
\Function{Top}{$A$}
  \State \Return $A[1]$
\EndFunction
\end{algorithmic}

这一简单操作是常数时间$O(1)$的。我们这里省略了对于空堆的错误处理。如果堆为空,我们可以返回一个错误。

\subsubsection{弹出堆顶元素}
\index{二叉堆!pop}

弹出操作比获取顶部元素要复杂一些。我们需要在移除顶部元素后,通过执行\textproc{Heapify}算法检查并恢复堆的性质。我们可以简单地给出下面的实现,但是它的性能并不好。

\begin{algorithmic}[1]
\Function{Pop-Slow}{$A$}
  \State $x \gets$ \Call{Top}{$A$}
  \State \Call{Remove}{$A$, 1}
  \If{$A$ is not empty}
    \State \Call{Heapify}{$A$, 1}
  \EndIf
  \State \Return $x$
\EndFunction
\end{algorithmic}

这一算法首先用$x$记录下顶部元素,然后将数组中的第一个元素删除,数组的长度减一。如果此后数组不为空,就从新的第一个元素开始执行一次\textproc{Heapify}。

从长度为$n$的数组中删除第一个元素需要线性时间$O(n)$。这是因为我们需要将所有剩余的元素依次向前移动一位。这一操作成为了整个算法的瓶颈,使得算法的复杂度升高了。

为了解决这一问题,我们可以交换数组中的第一个和最后一个元素,然后将数组的长度减一。

\begin{algorithmic}[1]
\Function{Pop}{$A$}
  \State $x \gets$ \Call{Top}{$A$}
  \State $n \gets$ \Call{Heap-Size}{$A$}
  \State \textproc{Exchange} $A[1] \leftrightarrow A[n]$
  \State \Call{Remove}{$A, n$}
  \If{$A$ is not empty}
    \State \Call{Heapify}{$A$, 1}
  \EndIf
  \State \Return $x$
\EndFunction
\end{algorithmic}

从数组的末尾删除最后一个元素仅需要常数时间$O(1)$,而\textproc{Heapify}算法的时间是$O(\lg n)$的。这样整体上弹出操作算法的复杂度为对数时间$O(\lg n)$。下面的C例子程序实现了这一算法\footnote{此程序并未删除最后一个元素,而是复用数组的最后一个cell来存储弹出的结果。}。

\lstset{language=C}
\begin{lstlisting}
Key pop(Key* a, int n, Less lt) {
    swap(a, 0, --n);
    heapify(a, 0, n, lt);
    return a[n];
}
\end{lstlisting}

\subsubsection{寻找top $k$个元素}
\index{二叉堆!top-k}

使用pop,可以很方便地找出一组值中的前$k$大个(或前$k$小)。我们可以构建一个最大堆,然后重复执行$k$次pop操作。

\begin{algorithmic}[1]
\Function{Top-k}{$A, k$}
  \State $R \gets \phi$
  \State \Call{Build-Heap}{$A$}
  \For{$i \gets 1$ to \textproc{Min}(k, |$A$|)}
    \State \textproc{Append}($R$, \Call{Pop}{$A$})
  \EndFor
  \State \Return $R$
\EndFunction
\end{algorithmic}

如果$k$超过了数组的长度,我们返回整个数组作为结果。因此上述实现中,我们使用最小值\textproc{Min}函数来决定循环的次数。

下面的Python例子程序实现了top-$k$算法:

\lstset{language=Python}
\begin{lstlisting}
def top_k(x, k, less_p = MIN_HEAP):
    build_heap(x, less_p)
    return [heap_pop(x, less_p) for _ in range(min(k, len(x)))]
\end{lstlisting}

\subsubsection{减小key值}
\index{二叉堆!decrease key}

堆可以用来实现带有优先级的队列。因此我们需要提供方法来更改堆中的key值。例如,我们在实际应用中,会增加一个任务的优先级来使得它能尽早执行。

这里我们给出在最小堆中减小key的结果,最大堆的相应操作为增加其中的key。图\ref{fig:decrease-key-1}和\ref{fig:decrease-key-2}描述了将最大堆中第9个节点从4增加到15的步骤。

\begin{figure}[htbp]
    \centering
    \subfloat[第9个节点的值为4。]{\includegraphics[scale=0.5]{img/decrease-key-a.ps}} \\
    \subfloat[将4增加到15,大于其父节点的值。]{\includegraphics[scale=0.5]{img/decrease-key-b.ps}} \\
    \subfloat[根据最大堆的性质,交换8和15。]{\includegraphics[scale=0.5]{img/decrease-key-c.ps}}
    \caption{增大最大堆中某个值的例子过程} \label{fig:decrease-key-1}
\end{figure}

\begin{figure}[htbp]
    \centering
    \subfloat[因为15大于父节点的值14,它们进行交换。此后因为15小于16,处理过程结束。]{\includegraphics[scale=0.5]{img/decrease-key-d.ps}}
    \caption{增大最大堆中某个值的例子过程} \label{fig:decrease-key-2}
\end{figure}

当最小堆中的某个值减小时,可能会违反堆的性质,新的key可能比它的祖先小。我们可以定义如下算法来恢复堆的性质。

\begin{algorithmic}[1]
\Function{Heap-Fix}{$A, i$}
  \While{$i>1 \land A[i] < A[$ \Call{Parent}{$i$} $]$}
    \State \textproc{Exchange} $A[i] \leftrightarrow A[$ \Call{Parent}{$i$} $]$
    \State $i \gets$  \Call{Parent}{$i$}
  \EndWhile
\EndFunction
\end{algorithmic}

这一算法不断比较当前节点和父节点的值,如果父节点较小,就进行交换。算法自底向上进行检查,直到到达根节点,或者发现父节点的值较小。

使用这一辅助算法,我们可以实现减小最小堆中key的操作。

\begin{algorithmic}[1]
\Function{Decrease-Key}{$A, i, k$}
  \If{$k < A[i]$}
    \State $A[i] \gets k$
    \State \Call{Heap-Fix}{$A, i$}
  \EndIf
\EndFunction
\end{algorithmic}

这一算法仅仅在新key比此前的值小时才有效。算法的性能是$O(\lg n)$的。下面的C例子程序实现了此算法。

\lstset{language=C}
\begin{lstlisting}
void heap_fix(Key* a, int i, Less lt) {
    while (i > 0 && lt(a[i], a[PARENT(i)])) {
        swap(a, i, PARENT(i));
        i = PARENT(i);
    }
}

void decrease_key(Key* a, int i, Key k, Less lt) {
    if (lt(k, a[i])) {
        a[i] = k;
        heap_fix(a, i, lt);
    }
}
\end{lstlisting}

\subsubsection{插入}
\index{二叉堆!插入}
\index{二叉堆!heap push}

插入可以用\textproc{Decrease-Key}来实现\cite{CLRS}。先构建一个key为$\infty$的新节点。根据最小堆的性质,新节点为数组中的最后一个元素。然后,我们将节点的key减小为待插入的值,再使用\textproc{Decrease-Key}恢复堆性质。

我们也可以直接使用\textproc{Heap-Fix}来实现插入。将待插入的元素直接附加到数组末尾,然后使用\textproc{Heap-Fix}自底向上恢复堆性质。

\begin{algorithmic}[1]
\Function{Heap-Push}{$A, k$}
  \State \Call{Append}{$A, k$}
  \State \Call{Heap-Fix}{$A, |A|$}
\EndFunction
\end{algorithmic}

下面的Python例子程序实现了堆插入算法。

\lstset{language=Python}
\begin{lstlisting}
def heap_insert(x, key, less_p = MIN_HEAP):
    i = len(x)
    x.append(key)
    heap_fix(x, i, less_p)
\end{lstlisting}

% ================================================================
%                 Heap sort
% ================================================================
\subsection{堆排序}
\label{heap-sort}
\index{堆排序}

堆排序是堆的一个有趣应用。根据堆的性质,可以很容易地从堆顶获取最小(或最大)元素。我们可以从待排序的元素构建一个堆,然后不断将最小元素弹出直到堆变空。

根据这一想法设计的算法如下:

\begin{algorithmic}[1]
\Function{Heap-Sort}{$A$}
  \State $R \gets \phi$
  \State \Call{Build-Heap}{$A$}
  \While{$A \neq \phi$}
    \State \textproc{Append}($R$, \Call{Heap-Pop}{$A$})
  \EndWhile
  \State \Return $R$
\EndFunction
\end{algorithmic}

下面的Python例子程序实现了这一定义。

\lstset{language=Python}
\begin{lstlisting}
def heap_sort(x, less_p = MIN_HEAP):
    res = []
    build_heap(x, less_p)
    while x!=[]:
        res.append(heap_pop(x, less_p))
    return res
\end{lstlisting}

若待排序的元素有$n$个,通过\textproc{Build-Heap}构建堆的复杂度是$O(n)$的。由于pop操作的复杂度为$O(\lg n)$,并且共执行了$n$次。因此堆排序的总体的复杂度为$O(n \lg n)$。由于我们使用了另外一个列表还存放排序结果,因此需要的空间为$O(n)$。

Robert. W. Floyd给出了一个堆排序的高效实现。思路是构建一个最大堆而不是最小堆。这样第一个元素就是最大的。接下来,将最大的元素和数组末尾的元素交换,这样最大元素就存储到了排序后的正确位置。而原来在末尾的元素变成了新的堆顶。这可能违反堆的性质,我们需要将堆的大小减一,然后执行\textproc{Heapify}恢复堆的性质。我们重复这一过程,直到堆中仅剩下一个元素。

\begin{algorithmic}[1]
\Function{Heap-Sort}{$A$}
  \State \Call{Build-Max-Heap}{$A$}
  \While{$|A| > 1$}
    \State \textproc{Exchange} $A[1] \leftrightarrow A[n]$
    \State $|A| \gets |A| - 1$
    \State \Call{Heapify}{$A, 1$}
  \EndWhile
\EndFunction
\end{algorithmic}

这一算法是in-place的,无需使用额外的空间来存储结果。下面的C例子程序实现了此算法。

\lstset{language=C}
\begin{lstlisting}
void heap_sort(Key* a, int n) {
    build_heap(a, n, notless);
    while(n > 1) {
        swap(a, 0, --n);
        heapify(a, 0, n, notless);
    }
}
\end{lstlisting}

\begin{Exercise}
\begin{itemize}
\item 考虑另外一种实现in-place堆排序的方法:第一步先从待排序数组构建一个最小堆$A$,此时,第一个元素$a_1$已经在正确的位置了。接下来,将剩余的元素$\{a_2, a_3, ..., a_n\}$当成一个新的堆,并从$a_2$开始执行\textproc{Heapify}。重复这一从左向右的步骤完成排序。下面的C语言代码实现了这一想法。这一方法正确么?如果正确,请给出证明,如果错误,请给出原因。
\lstset{language=C}
\begin{lstlisting}
void heap_sort(Key* a, int n) {
    build_heap(a, n, less);
    while(--n)
        heapify(++a, 0, n, less);
}
\end{lstlisting}

\item 由于同样的原因,我们可以通过自左向右执行$k$遍\textproc{Heapify}来实现in-place的top-$k$算法么?如下面的C语言例子代码所示:
\lstset{language=C}
\begin{lstlisting}
int tops(int k, Key* a, int n, Less lt) {
    build_heap(a, n, lt);
    for (k = MIN(k, n) - 1; k; --k)
        heapify(++a, 0, --n, lt);
    return k;
}
\end{lstlisting}
\end{itemize}
\end{Exercise}

% ================================================================
%                 Explicit binary heap
% ================================================================
\section{左偏堆和Skew堆―显式二叉堆}
\label{ebheap}

人们很自然会问:如果不使用数组表示的隐式二叉堆,有没有可能使用普通的二叉树来实现堆?

如果使用显式的二叉树作为堆的底层数据结构,我们必须解决一些问题。第一个问题是关于\textproc{Heap-Pop}和\textproc{Delete-Min}操作的。考虑图\ref{fig:lvr}所示的二叉树$(L, k, R)$,其中$L$、$k$、$R$分别表示左子树、key和右子树。

\begin{figure}[htbp]
    \centering
    \includegraphics[scale=0.8]{img/lvr.ps}
    \caption{二叉树,所有子节点中的元素都大于$k$。} \label{fig:lvr}
\end{figure}

如果$k$是一个最小堆的顶部元素,所有左右子树中的元素都小于$k$。$k$被弹出后,只剩下左右子树。我们需要把它们合并为一棵新树。由于合并后必须保持堆的性质,新的根节点必须仍保存剩余元素中的最小元素。

因为左右子树也都是符合堆性质的二叉树,我们可以立即给出两个特殊情况下的结果:

\[
merge(H_1, H_2) = \left \{
  \begin{array}
  {r@{\quad:\quad}l}
  H_2 & H_1 = \phi \\
  H_1 & H_2 = \phi \\
  ? & otherwise
  \end{array}
\right.
\]

其中$\phi$表示空堆。如果左右子树都不为空,因为它们都满足堆的性质,因此各自的根节点都保存了最小的元素。我们可以比较两棵树的根,选择较小的一个作为堆合并后的根。

举例来说,令$L = (A, x, B)$、$R = (A', y, B')$,其中$A$、$A'$、$B$、$B'$都是子树,如果$x < y$,$x$就将是新的根。我们或者可以保留$A$,然后递归地将$B$和$R$合并;或者保留$B$,然后递归地合并$A$和$R$。新的堆可以为下面之一:

\begin{itemize}
\item $(merge(A, R), x, B)$
\item $(A, x, merge(B, R))$
\end{itemize}

两个都是正确的结果,为了简单,我们可以总选择右侧的子树进行合并。左偏堆({\em Leftist} heap)就是基于这一思想实现的。

% ================================================================
%                 Definition
% ================================================================
\subsection{定义}
\index{左偏堆}
\index{Leftist heap}

使用左偏树实现的堆称为左偏堆。左偏树最早由C. A. Crane于1972年引入\cite{wiki-leftist-tree}。

\subsubsection{Rank(S-值)}
\index{左偏堆!rank}
\index{左偏堆!S-值}

左偏树中每个节点都定义了一个rank值(或称$S$值)。Rank被定义为到达最近的外部节点的距离。其中外部节点指空节点NIL,例如叶子节点的子节点就是外部节点。

如图\ref{fig:rank}所示,NIL的rank被定义为0。考虑根节点4,最近的叶子节点为8,所以根节点的rank为2。因为节点6和节点8都是叶子节点,所以它们的rank为1。虽然节点5则左子树不为空,但是它的右子树是空节点,因子rank值,也就是到达NIL的最短距离仍然为1。

\begin{figure}[htbp]
   \begin{center}
     \includegraphics[scale=0.5]{img/rank.ps}
     \caption{$rank(4) = 2$、$rank(6) = rank(8) = rank(5) = 1$} \label{fig:rank}
   \end{center}
\end{figure}

\subsubsection{左偏性质}

使用rank,我们可以定义合并时的策略:

\begin{itemize}
\item 总是合并右侧子树。记新的右侧子树的rank值为$r_r$;
\item 比较左右子树的rank值,记左子树的rank值为$r_l$,如果$r_l < r_r$,就交换左右子树。
\end{itemize}

我们称上面的合并策略为“左偏性质”。概括来说,在一棵左偏树中,到某个外部节点的最短距离总是在右侧。

左偏树总是趋向不平衡,但是它可以维护一条重要的性质,如下面的定理所述:

\begin{theorem}
若一棵左偏树$T$包含$n$各内部节点,从根节点到达最右侧的外部节点的路径上最多含有$\lfloor \log (n+1) \rfloor$各节点。
\end{theorem}

我们这里省略了此定理的证明,读者可以参考\cite{brono-book},\cite{TAOCP}来了解证明的过程。根据此定理,沿着这一路径进行操作的算法,都可以保证$O(\lg n)$的复杂度。

我们可以在二叉树定义的基础上增加一个rank值来定义左偏树。记非空的左偏树为$(r, k, L, R)$。下面的Haskell例子程序定义了左偏树。

\lstset{language=Haskell}
\begin{lstlisting}
data LHeap a = E -- Empty
             | Node Int a (LHeap a) (LHeap a) -- rank, element, left, right
\end{lstlisting}

我们定义空树的rank为0,否则,我们通过读取新增加的变量$r$来获得rank值。下面的$rank(H)$函数可以获取任一情况下的值。

\be
rank(H) = \left \{
  \begin{array}
  {r@{\quad:\quad}l}
  0 & H = \phi \\
  r & otherwise, H = (r, k, L, R)
  \end{array}
\right.
\ee

对应的Haskell例子程序如下:

\lstset{language=Haskell}
\begin{lstlisting}
rank E = 0
rank (Node r _ _ _) = r
\end{lstlisting}

方便起见,我们以后将$rank(H)$简记为$r_H$。

% ================================================================
%                 Merge
% ================================================================
\subsection{合并}
\index{左偏堆!合并}

为了实现合并操作,我们需要显定义一个算法用以比较左右子树的rank值,并适当地进行子树的交换。

\be
mk(k, A, B) = \left \{
  \begin{array}
  {r@{\quad:\quad}l}
  (r_A + 1, k, B, A) & r_A < r_B \\
  (r_B + 1, k, A, B) & otherwise
  \end{array}
\right.
\ee

这一函数接受三个参数,一个key和两棵子树$A$、$B$。如果$A$的rank较小,算法就用$B$作为左子树,$A$作为右子树来构建一棵较大的树。然后它将$A$的rank加一作为这棵新树的rank值;否则,如果$B$的rank较小,就用A作为左子树,$B$作为右子树。新树的rank值为$r_b + 1$。

由于构造新树的时候,我们在顶部增加了一个新的key。所以rank的值会增长1。

给定两个左偏堆$H_1$和$H_2$,记它们的key和左右子树分别为:$k_1, L_1, R_1$和$k_2, L_2, R_2$。下面的$merge(H_1, H_2)$函数定义了合并算法:

\be
merge(H_1, H_2) = \left \{
  \begin{array}
  {r@{\quad:\quad}l}
  H_2 & H_1 = \phi \\
  H_1 & H_2 = \phi \\
  mk(k_1, L_1, merge(R_1, H_2)) & k_1 < k_2 \\
  mk(k_2, L_2, merge(H_1, R_2)) & otherwise
  \end{array}
\right.
\ee

函数$merge$总是在右子树上进行递归调用,因此左偏的性质得以保持。这样就保证了算法的复杂度为$O(\lg n)$。

下面的Haskell例子代码实现了合并算法。

\lstset{language=Haskell}
\begin{lstlisting}
merge E h = h
merge h E = h
merge h1@(Node _ x l r) h2@(Node _ y l' r') =
    if x < y then makeNode x l (merge r h2)
    else makeNode y l' (merge h1 r')

makeNode x a b = if rank a < rank b then Node (rank a + 1) x b a
                 else Node (rank b + 1) x a b
\end{lstlisting}

\subsubsection{合并由数组表示的二叉堆}
\index{二叉堆!合并}

使用数组表示的二叉堆在大多数情况下速度都很快。并且很和现代计算机的cache技术配合良好。但是合并操作的算法复杂度却为线性时间$O(n)$。通常的实现是将两个数组连接起来,然后在连接后的结果上重新构建堆\cite{NIST}。

\begin{algorithmic}[1]
\Function{Merge-Heap}{$A, B$}
  \State $C \gets$ \Call{Concat}{$A, B$}
  \State \Call{Build-Heap}{$C$}
\EndFunction
\end{algorithmic}

% ================================================================
%                 Basic heap operations
% ================================================================
\subsection{基本堆操作}

Most of the basic heap operations can be implemented with $merge$
algorithm defined above.

\subsubsection{Top and pop}
\index{Leftist heap!top}
\index{Leftist heap!pop}
Because the smallest element is always held in root, it's trivial
to find the minimum value. It's constant $O(1)$ operation. Below
equation extracts the root from non-empty heap $H = (r, k, L, R)$.
The error handling for empty case is skipped here.

\be
top(H) = k
\ee

For pop operation, firstly, the top element is removed, then
left and right children are merged to a new heap.

\be
pop(H) = merge(L, R)
\ee

Because it calls $merge$ directly, the pop operation on Leftist heap is bound
to $O(\lg n)$.

\subsubsection{Insertion}
\index{Leftist heap!insertion}

To insert a new element, one solution is to create a single
leaf node with the element, and then merge this leaf node to
the existing Leftist tree.

\be
insert(H, k) = merge(H, (1, k, \phi, \phi))
\ee

It is $O(\lg n)$ algorithm since insertion also calls $merge$ directly.

There is a convenient way to build the Leftist heap from
a list. We can continuously insert the elements one by one
to the empty heap. This can be realized by folding.

\be
build(L) = fold(insert, \phi, L)
\ee

Figure \ref{fig:leftist-tree} shows one example Leftist tree
built in this way.

\begin{figure}[htbp]
   \begin{center}
   	  \includegraphics[scale=0.5]{img/leftist-tree.ps}
    \caption{A Leftist tree built from list $\{9, 4, 16, 7, 10, 2, 14, 3, 8, 1\}$.}
    \label{fig:leftist-tree}
   \end{center}
\end{figure}

The following example Haskell code gives reference implementation
for the Leftist tree operations.

\lstset{language=Haskell}
\begin{lstlisting}
insert h x = merge (Node 1 x E E) h

findMin (Node _ x _ _) = x

deleteMin (Node _ _ l r) = merge l r

fromList = foldl insert E
\end{lstlisting}

% ================================================================
%                 Heap sort
% ================================================================
\subsection{Heap sort by Leftist Heap}
\index{Leftist heap!heap sort}

With all the basic operations defined, it's straightforward to
implement heap sort. We can firstly turn the list into a Leftist
heap, then continuously extract the minimum
element from it.

\be
sort(L) = heapSort(build(L))
\ee

\be
heapSort(H) = \left \{
  \begin{array}
  {r@{\quad:\quad}l}
  \phi & H = \phi \\
  \{top(H)\} \cup heapSort(pop(H)) & otherwise
  \end{array}
\right.
\ee

Because pop is logarithm operation, and it is recursively called $n$ times,
this algorithm takes $O(n \lg n)$ time in total. The following Haskell
example program implements heap sort with Leftist tree.

\lstset{language=Haskell}
\begin{lstlisting}
heapSort = hsort . fromList where
    hsort E = []
    hsort h = (findMin h):(hsort $ deleteMin h)
\end{lstlisting} %$



% ================================================================
%                 Skew Heap
% ================================================================

\subsection{Skew heaps}
\label{skew-heap}
\index{Skew heap}

Leftist heap performs poor in some cases. Figure \ref{fig:unbalanced-leftist-tree}
shows one example. The Leftist tree is built by folding on
list $\{16, 14, 10, 8, 7, 9, 3, 2, 4, 1\}$.

\begin{figure}[htbp]
   \begin{center}
   	  \includegraphics[scale=0.3]{img/unbalanced-leftist-tree.ps}
    \caption{A very unbalanced Leftist tree build from list $\{16, 14, 10, 8, 7, 9, 3, 2, 4, 1\}$.}
    \label{fig:unbalanced-leftist-tree}
   \end{center}
\end{figure}

The binary tree almost turns to a linked-list. The worst case
happens when feed the ordered list to build Leftist tree. Because the
tree downgrades to linked-list, the performance drops from $O(\lg n)$
to $O(n)$.

Skew heap (or {\em self-adjusting heap}) simplifies Leftist heap realization
and can solve the performance issue\cite{wiki-skew-heap} \cite{self-adjusting-heaps}.

When construct the Leftist heap, we swap the left and right children during merge
if the rank on left side is less than the right side. This comparison-and-swap strategy
doesn't work when either sub tree has only one child. Because
in such case, the rank of the sub tree is always 1 no matter how
big it is. A `Brute-force' approach is to swap the left and right children
every time when merge. This idea leads to Skew heap.

\subsubsection{Definition of Skew heap}

Skew heap is the heap realized with Skew tree. Skew tree is a special
binary tree. The minimum element is stored in root. Every sub tree is
also a skew tree.

It needn't keep the rank (or $S$-value) field. We can reuse the
binary tree definition for Skew heap. The tree is either empty,
or in a pre-order form $(k, L, R)$. Below Haskell code defines
Skew heap like this.

\lstset{language=Haskell}
\begin{lstlisting}
data SHeap a = E -- Empty
             | Node a (SHeap a) (SHeap a) -- element, left, right
\end{lstlisting}

\subsubsection{Merge}
\index{Skew heap!merge}
\index{Skew heap!insertion}
\index{Skew heap!top}
\index{Skew heap!pop}

The merge algorithm tends to be very simple.
When merge two non-empty Skew
trees, we compare the roots, and pick the smaller
one as the new root, then the other tree contains the bigger
element is merged onto one sub tree, finally,
the tow children are swapped. Denote $H_1 = (k_1, L_1, R_1)$
and $H_2 =(k_2, L_2, R_2)$ if they are not empty.
if $k_1 < k_2$ for instance, select $k_1$ as the new root. We can
either merge $H_2$ to $L_1$, or merge $H_2$ to $R_1$.
Without loss of generality, let's merge to $R_1$.
And after swapping the two children, the final result
is $(k_1, merge(R_1, H_2), L_1)$. Take account of
edge cases, the merge algorithm is defined as the
following.

\be
merge(H_1, H_2) = \left \{
  \begin{array}
  {r@{\quad:\quad}l}
  H_1 & H_2 = \phi \\
  H_2 & H_1 = \phi \\
  (k_1, merge(R_1, H_2), L_1) & k_1 < k_2 \\
  (k_2, merge(H_1, R_2), L_2) & otherwise
  \end{array}
\right.
\ee

All the rest operations, including insert, top and pop are all
realized as same as the Leftist heap by using merge, except that
we needn't the rank any more.

Translating the above algorithm into Haskell yields the following
example program.

\lstset{language=Haskell}
\begin{lstlisting}
merge E h = h
merge h E = h
merge h1@(Node x l r) h2@(Node y l' r') =
    if x < y then Node x (merge r h2) l
    else Node y (merge h1 r') l'

insert h x = merge (Node x E E) h

findMin (Node x _ _) = x

deleteMin (Node _ l r) = merge l r
\end{lstlisting}

Different from the Leftist heap, if we feed ordered list to Skew heap, it can build a
fairly balanced binary tree as illustrated in figure \ref{fig:skew-tree}.

\begin{figure}[htbp]
   \begin{center}
   	  \includegraphics[scale=0.5]{img/skew-tree.ps}
    \caption{Skew tree is still balanced even the input is an ordered list $\{1, 2, ..., 10\}$.}
    \label{fig:skew-tree}
   \end{center}
\end{figure}


% ================================================================
%                 Splay Heap
% ================================================================

\section{Splay heap}
\label{splayheap}
\index{Splay heap}

The Leftist heap and Skew heap show the fact that it's quite possible to realize
heap data structure with explicit binary tree.
Skew heap gives one method to solve the tree balance problem. Splay heap
on the other hand, use another method to keep the tree balanced.

The binary trees used in Leftist heap and Skew heap
are not Binary Search tree (BST). If we turn the underground
data structure to binary search tree, the minimum(or maximum)
element is not root any more. It takes $O(\lg n)$ time
to find the minimum(or maximum) element.

Binary search tree becomes inefficient if it isn't well
balanced. Most operations degrade to $O(n)$ in the worst case.
Although red-black tree can be used to realize
binary heap, it's overkill. Splay tree provides a light weight
implementation with acceptable dynamic balancing result.


% ================================================================
%                 Definition
% ================================================================
\subsection{Definition}

Splay tree uses cache-like approach. It keeps rotating the current
access node close to the top, so that the node can be accessed fast
next time. It defines such kinds of operation as ``Splay''. For the
unbalanced binary search tree, after several splay operations, the
tree tends to be more and more balanced. Most basic operations of
Splay tree perform in amortized $O(\lg n)$ time. Splay tree was invented
by Daniel Dominic Sleator and Robert Endre Tarjan in 1985\cite{wiki-splay-tree}
\cite{self-adjusting-trees}.

\subsubsection{Splaying}
\index{Splay heap!splaying}

There are two methods to do splaying. The first one need deal
with many different cases, but can be implemented fairly easy with
pattern matching. The second one has a uniformed form, but the implementation
is complex.

Denote the node currently being accessed as $X$, the parent node as $P$,
and the grand parent node as $G$ (If there are).  There are 3 steps for
splaying. Each step contains 2 symmetric cases. For illustration
purpose, only one case is shown for each step.

\begin{itemize}
\item {\em Zig-zig step.} As shown in figure \ref{fig:zig-zig}, in this case,
$X$ and $P$ are children on the same side, either both on the left or on the right. By
rotating 2 times, $X$ becomes the new root.

\begin{figure}[htbp]
  \centering
  \subfloat[$X$ and $P$ are either left or right children.]{\includegraphics[scale=0.4]{img/zig-zig-a.ps}}
  \subfloat[$X$ becomes new root after rotating 2 times.]{\includegraphics[scale=0.4]{img/zig-zig-b.ps}}
  \caption{Zig-zig case.} \label{fig:zig-zig}
\end{figure}

\item {\em Zig-zag step.} As shown in figure \ref{fig:zig-zag}, in this
case, $X$ and $P$ are children on different sides. $X$ is on the left,
$P$ is on the right. Or $X$ is on the right, $P$ is on the left.
After rotation, $X$ becomes the new root, $P$ and $G$ are siblings.

\begin{figure}[htbp]
  \centering
  \subfloat[$X$ and $P$ are children on different sides.]{\includegraphics[scale=0.4]{img/zig-zag-a.ps}}
  \subfloat[$X$ becomes new root. $P$ and $G$ are siblings.]{\includegraphics[scale=0.4]{img/zig-zag-b.ps}}
  \caption{Zig-zag case.} \label{fig:zig-zag}
\end{figure}

\item {\em Zig step.} As shown in figure \ref{fig:zig}, in this case,
$P$ is the root, we rotate the tree, so that $X$ becomes new root.
This is the last step in splay operation.

\begin{figure}[htbp]
  \centering
  \subfloat[$P$ is the root.]{\includegraphics[scale=0.4]{img/zig-a.ps}}
  \subfloat[Rotate the tree to make $X$ be new root.]{\includegraphics[scale=0.4]{img/zig-b.ps}}
  \caption{Zig case.} \label{fig:zig}
\end{figure}

\end{itemize}

Although there are 6 different cases, they can be handled in the
environments support pattern matching. Denote the binary tree
in form $(L, k, R)$. When access key $Y$ in tree $T$, the splay
operation can be defined as below.

\be
splay(T, X) = \left \{
  \begin{array}
  {r@{\quad:\quad}l}
  (a, X, (b, P, (c, G, d))) & T = (((a, X, b), P, c), G, d), X = Y \\
  (((a, G, b), P, c), X, d) & T= (a, G, (b, P, (c, X, d))), X = Y \\
  ((a, P, b), X, (c, G, d)) & T = (a, P, (b, X, c), G, d), X = Y \\
  ((a, G, b), X, (c, P, d)) & T = (a, G, ((b, X, c), P, d)), X = Y \\
  (a, X, (b, P, c)) & T = ((a, X, b), P, c), X = Y \\
  ((a, P, b), X, c) & T = (a, P, (b, X, c)), X = Y \\
  T &  otherwise
  \end{array}
\right.
\ee

The first two clauses handle the 'zig-zig' cases; the next two
clauses handle the 'zig-zag' cases; the last two clauses handle
the zig cases. The tree aren't changed for all other situations.

The following Haskell program implements this splay function.

\lstset{language=Haskell}
\begin{lstlisting}
data STree a = E -- Empty
             | Node (STree a) a (STree a) -- left, key, right

-- zig-zig
splay t@(Node (Node (Node a x b) p c) g d) y =
    if x == y then Node a x (Node b p (Node c g d)) else t
splay t@(Node a g (Node b p (Node c x d))) y =
    if x == y then Node (Node (Node a g b) p c) x d else t
-- zig-zag
splay t@(Node (Node a p (Node b x c)) g d) y =
    if x == y then Node (Node a p b) x (Node c g d) else t
splay t@(Node a g (Node (Node b x c) p d)) y =
    if x == y then Node (Node a g b) x (Node c p d) else t
-- zig
splay t@(Node (Node a x b) p c) y = if x == y then Node a x (Node b p c) else t
splay t@(Node a p (Node b x c)) y = if x == y then Node (Node a p b) x c else t
-- otherwise
splay t _ = t
\end{lstlisting}

With splay operation defined, every time when insert a new key,
we call the splay function to adjust the tree.
If the tree is empty, the result is a leaf; otherwise we compare this key
with the root, if it is less than the root, we recursively insert it into
the left child, and perform splaying after that; else the key is inserted
into the right child.

\be
insert(T, x) = \left \{
  \begin{array}
  {r@{\quad:\quad}l}
  (\phi, x, \phi) & T = \phi \\
  splay((insert(L, x), k, R), x) & T = (L, k, R), x < k \\
  splay(L, k, insert(R, x)) & otherwise
  \end{array}
  \right.
\ee

The following Haskell program implements this insertion algorithm.

\lstset{language=Haskell}
\begin{lstlisting}
insert E y = Node E y E
insert (Node l x r) y
    | x > y     = splay (Node (insert l y) x r) y
    | otherwise = splay (Node l x (insert r y)) y
\end{lstlisting}

Figure \ref{fig:splay-result} shows the result of using this function.
It inserts the ordered elements $\{1, 2, ..., 10\}$
one by one to the empty tree. This would build a very poor result
which downgrade to linked-list with normal binary search tree.
The splay method creates more balanced result.

\begin{figure}[htbp]
  \centering
  \includegraphics[scale=0.5]{img/splay-tree.ps}
  \caption{Splaying helps improving the balance.}
  \label{fig:splay-result}
\end{figure}

Okasaki found a simple rule for Splaying \cite{okasaki-book}.
Whenever we follow
two left branches, or two right branches continuously, we rotate
the two nodes.

Based on this rule, splaying can be realized in such a way.
When we access node for a key $x$ (can be during the process of
inserting a node, or looking up a node, or deleting a node), if
we traverse two left branches or two right branches, we
partition the tree in two parts $L$ and $R$, where $L$ contains all
nodes smaller than $x$, and $R$ contains all the rest.
We can then create a new tree (for instance in insertion),
with $x$ as the root, $L$ as the left child, and $R$ being the right child.
The partition process is recursive, because it will splay
its children as well.

\be
partition(T, p) = \left \{
  \begin{array}
  {r@{\quad:\quad}l}
  (\phi, \phi) & T = \phi \\
  (T, \phi) & T = (L, k, R) \land R = \phi \\
  ((L, k, L'), k', A, B) & \begin{array}{l} \\
                             T = (L, k, (L', k', R')) \\
                             k < p, k' < p \\
                             (A, B) = partition(R', p)
                           \end{array} \\
  ((L, k, A), (B, k', R')) & \begin{array}{l} \\
                               T = (L, K, (L', k', R')) \\
                               k < p \leq k' \\
                               (A, B) = partition(L', p) \\ \\
                             \end{array} \\
  (\phi, T) & T = (L, k, R) \land L = \phi \\
  (A, (L', k', (R', k, R)) & \begin{array}{l} \\
                               T = ((L', k', R'), k, R) \\
                               p \leq k, p \leq k' \\
                               (A, B) = partition(L', p)
                             \end{array} \\
  ((L', k', A), (B, k, R)) & \begin{array}{l} \\
                               T = ((L', k', R'), k, R) \\
                               k' \leq p \leq k \\
                               (A, B) = partition(R', p)
                             \end{array}
  \end{array}
  \right.
\ee

Function $partition(T, p)$ takes a tree $T$, and a pivot $p$ as arguments.
The first clause is edge case. The partition result for empty is
a pair of empty left and right trees. Otherwise, denote the tree
as $(L, k, R)$. we need compare the pivot $p$ and the root $k$.
If $k < p$, there are two sub-cases. one is trivial case that
$R$ is empty. According to the property of binary search tree,
All elements are less than $p$, so the result pair is $(T, \phi)$;
For the other case, $R = (L', k', R')$, we need further compare
$k'$ with the pivot $p$. If $k' < p$ is also true, we recursively
partition $R'$ with the pivot, all the elements less than $p$ in
$R'$ is held in tree $A$, and the rest is in tree $B$. The
result pair can be composed with two trees, one is $((L, k, L'), k', A)$;
the other is $B$. If the key of the right sub tree is not less than
the pivot. we recursively partition $L'$ with the pivot to give
the intermediate pair $(A, B)$, the final pair trees can be
composed with $(L, k, A)$ and $(B, k', R')$. There is a symmetric
cases for $p \leq k$. They are handled in the last three clauses.

Translating the above algorithm into Haskell yields the following partition
program.

\begin{lstlisting}
partition E _ = (E, E)
partition t@(Node l x r) y
    | x < y =
        case r of
          E -> (t, E)
          Node l' x' r' ->
              if x' < y then
                  let (small, big) = partition r' y in
                  (Node (Node l x l') x' small, big)
              else
                  let (small, big) = partition l' y in
                  (Node l x small, Node big x' r')
    | otherwise =
        case l of
          E -> (E, t)
          Node l' x' r' ->
              if y < x' then
                  let (small, big) = partition l' y in
                  (small, Node l' x' (Node r' x r))
              else
                  let (small, big) = partition r' y in
                  (Node l' x' small, Node big x r)
\end{lstlisting}

% ================================================================
%                 Basic heap operations
% ================================================================
\index{Splay heap!insertion}
Alternatively, insertion can be realized with $partition$ algorithm.
When insert a new element $k$ into
the splay heap $T$, we can first partition the heap into two trees, $L$ and $R$. Where
$L$ contains all nodes smaller than $k$, and $R$ contains the rest.
We then construct a new node, with $k$ as the root and $L$, $R$ as the children.

\be
insert(T, k) = (L, k, R), (L, R) = partition(T, k)
\ee

The corresponding Haskell example program is as the following.

\lstset{language=Haskell}
\begin{lstlisting}
insert t x = Node small x big where (small, big) = partition t x
\end{lstlisting}

\subsubsection{Top and pop}
\index{Splay heap!top}
\index{Splay heap!pop}
Since splay tree is just a special binary search tree, the minimum
element is stored in the left most node. We need keep traversing
the left child to realize the top operation. Denote the none empty
tree $T=(L, k, R)$, the $top(T)$ function can be defined as below.

\be
top(T) = \left \{
  \begin{array}
  {r@{\quad:\quad}l}
  k & L = \phi \\
  top(L) & otherwise
  \end{array}
  \right.
\ee

This is exactly the $min(T)$ algorithm for binary search tree.

For pop operation, the algorithm need remove the minimum element from the
tree. Whenever there
are two left nodes traversed, the splaying operation should be performed.

\be
pop(T) = \left \{
  \begin{array}
  {r@{\quad:\quad}l}
  R & T = (\phi, k, R) \\
  (R', k, R) & T = ((\phi, k', R') k, R) \\
  (pop(L'), k', (R', k, R)) & T = ((L', k', R'), k, R)
  \end{array}
  \right.
\ee

Note that the third clause performs splaying without explicitly call
the $partition$ function. It utilizes the property of binary
search tree directly.

Both the top and pop algorithms are bound to $O(\lg n)$ time because
the splay tree is balanced.

The following Haskell example programs implement the top and pop
operations.

\lstset{language=Haskell}
\begin{lstlisting}
findMin (Node E x _) = x
findMin (Node l x _) = findMin l

deleteMin (Node E x r) = r
deleteMin (Node (Node E x' r') x r) = Node r' x r
deleteMin (Node (Node l' x' r') x r) = Node (deleteMin l') x' (Node r' x r)
\end{lstlisting}

\subsubsection{Merge}
\index{Splay heap!merge}
Merge is another basic operation for heaps as it is widely used in Graph algorithms. By using the $partition$ algorithm, merge can be realized in $O(\lg n)$ time.

When merging two splay trees, for non-trivial case, we can take the root of the first tree as the new root, then partition the second tree with this new root as the pivot. After that we recursively merge
the children of the first tree to the partition result. This algorithm is defined as the following.

\be
merge(T_1, T_2) = \left \{
  \begin{array}
  {r@{\quad:\quad}l}
  T_2 & T_1 = \phi \\
  (merge(L, A), k, merge(R, B)) & T_1 = (L, k, R), (A, B) = partition(T_2, k)
  \end{array}
  \right.
\ee

If the first heap is empty, the result is definitely the second heap. Otherwise,
denote the first splay heap as $(L, k, R)$, we partition $T_2$ with $k$ as the
pivot to yield $(A, B)$, where $A$ contains all the elements in $T_2$ which are
less than $k$, and $B$ holds the rest. We next recursively merge $A$ with $L$;
and merge $B$ with $R$ as the new children for $T_1$.

Translating the definition to Haskell gives the following example program.

\lstset{language=Haskell}
\begin{lstlisting}
merge E t = t
merge (Node l x r) t = Node (merge l l') x (merge r r')
    where (l', r') = partition t x
\end{lstlisting}

% ================================================================
%                 Heap sort
% ================================================================
\subsection{Heap sort}

Since the internal implementation of the Splay heap is completely
transparent to the heap interface, the heap sort algorithm can
be reused. It means that the heap sort algorithm is generic no
matter what the underground data structure is.

% ================================================================
%                 Short summary
% ================================================================
\section{Notes and short summary}

In this chapter, we define binary heap more general
so that as long as the heap property is maintained, all binary
representation of data structures can be used to implement binary heap.

This definition doesn't limit to the popular array based binary
heap, but also extends to the explicit binary heaps including Leftist
heap, Skew heap and Splay heap. The array based binary heap
is particularly convenient for the imperative implementation
because it intensely uses random index access which can be mapped to
a completely binary tree. It's hard to find directly functional
counterpart in this way.

However, by using explicit binary tree, functional implementation
can be achieved, most of them have $O(\lg n)$ worst case
performance, and some of them even reach $O(1)$ amortize time.
Okasaki in \cite{okasaki-book} shows detailed analysis of these data
structures.

In this chapter, only purely functional realization for Leftist heap,
Skew heap, and Splay heap are explained, they can all be realized
in imperative approaches.

It's very natural to extend the concept from binary tree to
$k$-ary ($k$-way) tree, which leads to other useful heaps such as
Binomial heap, Fibonacci heap and pairing heap. They are introduced
in the next chapter.

% ================================================================
%                 Exercise
% ================================================================
\begin{Exercise}
\begin{itemize}
\item Realize the imperative Leftist heap, Skew heap, and Splay heap.
\end{itemize}
\end{Exercise}


\begin{thebibliography}{99}

\bibitem{CLRS}
Thomas H. Cormen, Charles E. Leiserson, Ronald L. Rivest and Clifford Stein. ``Introduction to Algorithms, Second Edition''. The MIT Press, 2001. ISBN: 0262032937.

\bibitem{wiki-heap}
Heap (data structure), Wikipedia. http://en.wikipedia.org/wiki/Heap\_(data\_structure)

\bibitem{wiki-heapsort}
Heapsort, Wikipedia. http://en.wikipedia.org/wiki/Heapsort

\bibitem{okasaki-book}
Chris Okasaki. ``Purely Functional Data Structures''. Cambridge university press, (July 1, 1999), ISBN-13: 978-0521663502

\bibitem{rosetta-heapsort}
Sorting algorithms/Heapsort. Rosetta Code. http://rosettacode.org/wiki/Sorting\_algorithms/Heapsort

\bibitem{wiki-leftist-tree}
Leftist Tree, Wikipedia. http://en.wikipedia.org/wiki/Leftist\_tree

\bibitem{brono-book}
Bruno R. Preiss. Data Structures and Algorithms with Object-Oriented Design Patterns in Java. http://www.brpreiss.com/books/opus5/index.html

\bibitem{TAOCP}
Donald E. Knuth. ``The Art of Computer Programming. Volume 3: Sorting and Searching.''. Addison-Wesley Professional;
2nd Edition (October 15, 1998). ISBN-13: 978-0201485417. Section 5.2.3 and 6.2.3

\bibitem{wiki-skew-heap}
Skew heap, Wikipedia. http://en.wikipedia.org/wiki/Skew\_heap

\bibitem{self-adjusting-heaps}
Sleator, Daniel Dominic; Jarjan, Robert Endre. ``Self-adjusting heaps'' SIAM Journal on Computing 15(1):52-69. doi:10.1137/0215004 ISSN 00975397 (1986)

\bibitem{wiki-splay-tree}
Splay tree, Wikipedia. http://en.wikipedia.org/wiki/Splay\_tree

\bibitem{self-adjusting-trees}
Sleator, Daniel D.; Tarjan, Robert E. (1985), ``Self-Adjusting Binary Search Trees'', Journal of the ACM 32(3):652 - 686, doi: 10.1145/3828.3835

\bibitem{NIST}
NIST, ``binary heap''. http://xw2k.nist.gov/dads//HTML/binaryheap.html

\end{thebibliography}

\ifx\wholebook\relax \else
\end{document}
\fi
