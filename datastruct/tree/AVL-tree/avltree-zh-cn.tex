\ifx\wholebook\relax \else
% ------------------------

\documentclass[UTF8]{article}
%------------------- Other types of document example ------------------------
%
%\documentclass[twocolumn]{IEEEtran-new}
%\documentclass[12pt,twoside,draft]{IEEEtran}
%\documentstyle[9pt,twocolumn,technote,twoside]{IEEEtran}
%
%-----------------------------------------------------------------------------
%
% loading packages
%

\RequirePackage{ifpdf}
\RequirePackage{ifxetex}

%
%
\ifpdf
  \RequirePackage[pdftex,%
       bookmarksnumbered,%
              colorlinks,%
          linkcolor=blue,%
              hyperindex,%
        plainpages=false,%
       pdfstartview=FitH]{hyperref}
\else\ifxetex
  \RequirePackage[bookmarksnumbered,%
               colorlinks,%
           linkcolor=blue,%
               hyperindex,%
         plainpages=false,%
        pdfstartview=FitH]{hyperref}
\else
  \RequirePackage[dvipdfm,%
        bookmarksnumbered,%
               colorlinks,%
           linkcolor=blue,%
               hyperindex,%
         plainpages=false,%
        pdfstartview=FitH]{hyperref}
\fi\fi
%\usepackage{hyperref}

% other packages
%--------------------------------------------------------------------------
\usepackage{graphicx, color}
\usepackage{subfig}
\usepackage{tikz}
\usetikzlibrary{matrix,positioning}

\usepackage{amsmath, amsthm, amssymb} % for math
\usepackage{exercise} % for exercise
\usepackage{import} % for nested input

%
% for programming
%
\usepackage{verbatim}
\usepackage{listings}
%\usepackage{algorithmic} %old version; we can use algorithmicx instead
\usepackage{algorithm}
\usepackage[noend]{algpseudocode} %for pseudo code, include algorithmicsx automatically
\usepackage{appendix}
\usepackage{makeidx} % for index support
\usepackage{titlesec}

\usepackage[cm-default]{fontspec}
\usepackage{xunicode}

% detect and select Chinese font
% ------------------------------
% the following cmd can list all availabe Chinese fonts in host.
% fc-list :lang=zh
\def\myfont{STHeiti}  % Under Mac OS X
\def\linuxfallback{WenQuanYi Micro Hei} % Under Linux
\def\winfallback{SimSun} % Under Windows
\suppressfontnotfounderror1 % Avoid setting exit code (error level) to break make process
\count255=\interactionmode
\batchmode
\font\foo="\myfont"\space at 10pt
\ifx\foo\nullfont
  \font\foo = "\linuxfallback"\space at 10pt
  \ifx\foo\nullfont
    \font\foo = "\winfallback"\space at 10pt
    \ifx\foo\nullfont
      \errorstopmode
      \errmessage{no suitable Chinese font found}
    \else
      \let\myfont=\winfallback % Windows
    \fi
  \else
    \let\myfont=\linuxfallback % Linux
  \fi
\fi
\interactionmode=\count255
\setmainfont[Mapping=tex-text]{\myfont}

\XeTeXlinebreaklocale "zh"  % to solve the line breaking issue
\XeTeXlinebreakskip = 0pt plus 1pt minus 0.1pt

\titleformat{\paragraph}
{\normalfont\normalsize\bfseries}{\theparagraph}{1em}{}
\titlespacing*{\paragraph}
{0pt}{3.25ex plus 1ex minus .2ex}{1.5ex plus .2ex}

\lstdefinelanguage{Smalltalk}{
  morekeywords={self,super,true,false,nil,thisContext}, % This is overkill
  morestring=[d]',
  morecomment=[s]{"}{"},
  alsoletter={\#:},
  escapechar={!},
  literate=
    {BANG}{!}1
    {UNDERSCORE}{\_}1
    {\\st}{Smalltalk}9 % convenience -- in case \st occurs in code
    % {'}{{\textquotesingle}}1 % replaced by upquote=true in \lstset
    {_}{{$\leftarrow$}}1
    {>>>}{{\sep}}1
    {^}{{$\uparrow$}}1
    {~}{{$\sim$}}1
    {-}{{\sf -\hspace{-0.13em}-}}1  % the goal is to make - the same width as +
    %{+}{\raisebox{0.08ex}{+}}1		% and to raise + off the baseline to match -
    {-->}{{\quad$\longrightarrow$\quad}}3
	, % Don't forget the comma at the end!
  tabsize=2
}[keywords,comments,strings]

% for better Haskell code outlook
\lstdefinelanguage{Haskell}{
  basicstyle=\small\ttfamily,
  flexiblecolumns=false,
  basewidth={0.5em,0.45em},
  literate={+}{{$+$}}1 {/}{{$/$}}1 {*}{{$*$}}1 {=}{{$=$}}1
           {>}{{$>$}}1 {<}{{$<$}}1 {\\}{{$\lambda$}}1
           {\\\\}{{\char`\\\char`\\}}1
           {->}{{$\rightarrow$}}2 {>=}{{$\geq$}}2 {<-}{{$\leftarrow$}}2
           {<=}{{$\leq$}}2 {=>}{{$\Rightarrow$}}2
           {\ .}{{$\circ$}}2 {\ .\ }{{$\circ$}}2
           {>>}{{>>}}2 {>>=}{{>>=}}2
           {|}{{$\mid$}}1
}[keywords,comments,strings]

\lstloadlanguages{C, C++, Lisp, Haskell, Python, Smalltalk}

\lstset{
  showstringspaces = false
}

% ======================================================================

\def\BibTeX{{\rm B\kern-.05em{\sc i\kern-.025em b}\kern-.08em
    T\kern-.1667em\lower.7ex\hbox{E}\kern-.125emX}}

%
% mathematics
%
\newcommand{\be}{\begin{equation}}
\newcommand{\ee}{\end{equation}}
\newcommand{\bmat}[1]{\left( \begin{array}{#1} }
\newcommand{\emat}{\end{array} \right) }
\newcommand{\VEC}[1]{\mbox{\boldmath $#1$}}

% numbered equation array
\newcommand{\bea}{\begin{eqnarray}}
\newcommand{\eea}{\end{eqnarray}}

% equation array not numbered
\newcommand{\bean}{\begin{eqnarray*}}
\newcommand{\eean}{\end{eqnarray*}}

\newtheorem{theorem}{Theorem}[section]
\newtheorem{lemma}[theorem]{Lemma}
\newtheorem{proposition}[theorem]{Proposition}
\newtheorem{corollary}[theorem]{Corollary}


\setcounter{page}{1}

\begin{document}

%--------------------------

% ================================================================
%                 COVER PAGE
% ================================================================

\title{AVL树}

\author{刘新宇
\thanks{{\bfseries 刘新宇} \newline
  Email: liuxinyu95@gmail.com \newline}
  }

\maketitle
\fi

\markboth{AVL树}{初等算法}

\ifx\wholebook\relax
\chapter{AVL树}
\numberwithin{Exercise}{chapter}
\fi

% ================================================================
%                 Introduction
% ================================================================
\section{简介}
\label{introduction} \index{AVL树}

\subsection{度量树的平衡性}

除了红黑树,还有没有其他自平衡二叉树呢?为了度量一棵二叉树的平衡,我们可以比较它左右分支的高度差,如果差很大,则说明树不平衡。定义一棵树的高度差如下:

\be
  \delta(T) = |L| - |R|
\ee

其中$|T|$代表树$T$的高度,$L$和$R$分别代表左右分支。

若$\delta(T) = 0$,说明树是平衡的。例如,一个高度为$h$的完全二叉树有$n = 2^h-1$个节点。除了叶子节点外,所有节点都含有两个非空的分支。完全二叉树就满足$\delta(T)=0$。另外一个特殊的例子是空树:$\delta(\phi) = 0$。通常$\delta(T)$的绝对值越小,说明树越平衡。

我们定义$\delta(T)$为一棵二叉树的\underline{平衡因子}。

% ================================================================
% Definition
% ================================================================
\section{AVL树的定义}
\index{AVL树!定义}

如果一棵二叉搜索树的所有子树都满足如下条件,我们称之为AVL树。

\be
  |\delta(T)| \leq 1
\ee

AVL树中所有子树平衡因子的绝对值都不大于1,只可能是-1、0,或1这三个值。图\ref{fig:avl-example}给出了一个AVL树的例子。

\begin{figure}[htbp]
   \centering
   \includegraphics[scale=0.5]{img/avl-example.ps}
   \caption{AVL树的例子} \label{fig:avl-example}
\end{figure}

为什么AVL树能保证树的平衡性呢?或者说为什么这个定义能保证一棵有$n$个节点的树的高度为$O(\lg n)$?我们可以用下面的方法来证明这一事实。

对于一棵高为$h$的AVL树,它的节点数目并不是一个固定的值。当它是一棵完全二叉树时,含有的节点数目最多,为$2^h-1$。那么它最少包含多少节点呢?我们记函数$N(h)$代表高度为$h$时,AVL树含有最少的节点数。对于简单的情况,我们可以立即给出$N(h)$的值:

\begin{itemize}
\item 空树,$h=0$,$N(0)=0$;
\item 只有一个根节点的树,$h=1$,$N(1)=1$;
\end{itemize}

一般情况下$N(h)$是怎样的呢?图\ref{fig:N-h-relation}中给出了一个高度为$h$的AVL树$T$。它包含三部份:根节点和左右两个分支$L$与$R$。树的高度和子树高度之间满足下面的关系:

\be
  h= max(|L|, |R|) + 1
\ee

因此,必然存在一个子树的高度为$h-1$。根据AVL树的定义,我们有 $||L| -|R|| \leq 1$。所以另外一棵子树的高度不会小于$h-2$。而$T$所包含的节点数为两个子树的节点数再加1(1个根节点)。于是我们得到下面的递归关系:

\be
  N(h) = N(h-1) + N(h-2) + 1
  \label{eq:Fibonacci-like}
\ee

\begin{figure}[htbp]
   \centering
   \includegraphics[scale=0.5]{img/Nh-lvr.ps}
   \caption{高度为$h$的AVL树,其中一个分支为$h-1$,另外一个分支的高度不小于$h-2$} \label{fig:N-h-relation}
\end{figure}

这一递归形式让我们联想起著名的斐波那契(Fibonacci)数列。如果定义$N'(h) = N(h)+1$,我们就可以将(\ref{eq:Fibonacci-like})转换成斐波那契数列。

\be
  N'(h) = N'(h-1) + N'(h-2)
\ee

\begin{lemma}
\label{lemma:N-phi}
定理:若$N(h)$表示高为$h$的AVL树的节点数目最小值,令$N'(h) = N(h) + 1$,则:

\be
  N'(h) \geq \phi^h
\ee

其中$\phi = \frac{\sqrt{5}+1}{2}$,通常被称为黄金分割比。
\end{lemma}

\begin{proof}
证明:使用数学归纳法。对于起始情况,我们有:
\begin{itemize}
\item $h=0$, $N'(0) = 1 \geq \phi^0 = 1$
\item $h=1$, $N'(1) = 2 \geq \phi^1 = 1.618...$
\end{itemize}

对于递推情况,设$N'(h) \geq \phi^h$。
\[
  \begin{array}{lll}
  N'(h+1) & = N'(h) + N'(h-1) & \{Fibonacci\} \\
          & \geq \phi^h + \phi^{h-1} & \\
          & = \phi^{h-1}(\phi + 1) & \{\phi + 1 = \phi^2 = \frac{\sqrt{5}+3}{2}\} \\
          & = \phi^{h+1}
 \end{array}
\]
\end{proof}

由定理\ref{lemma:N-phi},我们立即得到下面的结果:

\be
  h \leq log_{\phi}(n+1) = log_{\phi}(2) \cdot \lg (n+1) \approx 1.44 \lg (n+1)
  \label{eq:AVL-height}
\ee

这一不等式说明AVL树的高度为$O(\lg n)$,从而保证了平衡性。

在树的基本操作中,插入和删除会改变树的结构。如果由此导致平衡因子的绝对值超过1,就需要通过修复使得$|\delta|$恢复到1以内。常见的修复方法是使用树旋转。受到Okasaki在红黑树\cite{okasaki}中的思路启发,本章中,我们介绍一种模式匹配(pattern matching)方法。这种“改变―恢复”的操作,使得AVL树成为了一种自平衡二叉树。作为比较,本章同样也给出命令式的AVL树算法。

平衡因子$\delta$显然可以通过递归求出。另外一种方法是在每个节点中保存一分平衡因子的值,如果树被改变,我们只要更新这个值就可以了。这一方法不需要每次都计算。

根据这一思路,我们在二叉搜索树的定义中增加一个$\delta$变量,如下面的C++代码所示\footnote{有些实现不保存$\delta$,取而代之保存树的高度,如\cite{py-avl}。}。

\lstset{language=C++}
\begin{lstlisting}
template <class T>
struct node{
  int delta;
  T key;
  node* left;
  node* right;
  node* parent;
};
\end{lstlisting}

某些纯函数式实现使用不同的constructor来保存平衡因子$\delta$。例如在\cite{hackage}中,定义了4个不同的constructor:\texttt{E}、\texttt{N}、\texttt{P}、\texttt{Z}。其中,\texttt{E}代表空树$\phi$;\texttt{N}代表平衡因子为-1;\texttt{P}代表平衡因子为1;\texttt{Z}代表平衡因子为0。

本章中,我们直接在节点中保存平衡因子的值。

\lstset{language=Haskell}
\begin{lstlisting}
data AVLTree a = Empty
               | Br (AVLTree a) a (AVLTree a) Int
\end{lstlisting}

不改变树的操作,包括查找,寻找最大、最小值和二叉搜索树完全一样。我们略过它们,仅仅关注哪些会改变树结构的操作。

% ================================================================
%                 Insertion
% ================================================================
\section{插入}
\index{AVL树!插入}

在AVL树中插入一个新元素可能造成AVL树的平衡性被破坏,平衡因子$\delta$的绝对值有可能超过1。为了恢复平衡,可以根据不同的插入情形进行树的旋转操作。大多数命令式实现采用这种方法。

另外一种方法很像Okasaki在红黑树实现中使用的模式匹配方法。它的特点是简单直观。

向AVL树中插入一个新key,根节点的平衡因子的\underline{变化}会在$[-1, 1]$之间\footnote{注意:这里不是说平衡因子的值在$[-1, 1]$内,而是说它的变化在这个范围内。},树的高度最多增加1。我们需要递归地使用这一信息来更新其他层级上的平衡因子$\delta$。定义插入算法的结果为一对值$(T', \Delta H)$,其中$T'$为插入后的新树,$\Delta H$为树高度的增加值。令函数$first(pair)$可以取得一对值中的第一个元素。我们可以在二叉搜索树的插入算法上进行改动,定义AVL树的插入操作:

\be
insert(T, k) = first(ins(T, k))
\ee

其中

\be
ins(T, k) = \left \{
  \begin{array}
  {r@{\quad:\quad}l}
  ((\phi, k, \phi, 0), 1) & T = \phi \\
  tree(ins(L, k), k', (R, 0), \Delta) & k < k' \\
  tree((L, 0), k', ins(R, k), \Delta) & otherwise
  \end{array}
\right.
\label{eq:ins}
\ee

$L$、$R$、$k'$、$\Delta$的定义如下,它们分别表示左右子树,key和平衡因子。

\[
  \begin{array}{l}
  L = left(T) \\
  R = right(T) \\
  k' = key(T) \\
  \Delta = \delta(T)
  \end{array}
\]

向AVL树$T$中插入一个新key $k$时,如果树为空,结果为一个叶子节点,节点的key为$k$,平衡因子为0。树的高度增加1。

否则,如果$T$不为空,我们需要比较根节点的key k'和待插入key k的大小。如果$k$小于根节点的key,我们将其递归插入左子树,否则将其插入右子树。

我们前面定义了递归插入的结果为一对值,例如$(L', \Delta H_l)$。我们需要对插入的结果调整平衡,并更新高度的增加值。函数$tree()$被定义用来完成这一任务。它接受4个参数:$(L', \Delta H_l)$、$k'$、$(R', \Delta H_r)$和$\Delta$。这一函数的运算结果记为$(T', \Delta H)$。其中,$T'$为调整平衡后的树,$\Delta H$是树高度的增加值,定义如下:

\be
  \Delta H = |T'| - |T|
\ee

它可以进一步分解为4种情况。

\be
\begin{array}{rl}
  \Delta H & = |T'| - |T| \\
              & = 1 + max(|R'|, |L'|) - (1 + max(|R|, |L|)) \\
              & = max(|R'|, |L'|) - max(|R|, |L|) \\
              & = \left \{
                  \begin{array}{r@{\quad:\quad}l}
                  \Delta H_r & \Delta \geq 0 \land \Delta' \geq 0 \\
                  \Delta + \Delta H_r & \Delta \leq 0 \land \Delta' \geq 0 \\
                  \Delta H_l - \Delta & \Delta \geq 0 \land \Delta' \leq 0 \\
                  \Delta H_l & otherwise
                  \end{array} \right .
\end{array}
\ee

由于一次插入操作不可能同时增加左右分支的高度,因此我们可以做上述的情况分解。根据定义,平衡因子等于右子树的高度减去左子树的高度。这4种情况可以分别解释如下:

\begin{itemize}
\item 如果$\Delta \geq 0$并且$\Delta' \geq 0$。这说明在插入前后,右子树的高度都不小于左子树的高度。因子整个树高度的增加,全部“贡献”自右子树高度的变化$\Delta H_r$;

\item 如果$\Delta \leq 0$,说明在插入前,左子树的高度不小于右子树。但是插入后$\Delta' \geq 0$,说明右子树的高度由于插入操作增加了,而左子树的高度保持不变($|L'|=|L|$)。所以高度的增加为:
\[
\begin{array}{rll}
\Delta H & = max(|R'|, |L'|) - max (|R|, |L|) & \{\Delta \leq 0 \land \Delta' \geq 0 \}\\
         & = |R'|-|L| & \{|L|=|L'| \}\\
         & = |R|+\Delta H_r - |L| & \\
         & = \Delta + \Delta H_r &
\end{array}
\]

\item 如果$\Delta \geq 0 \land \Delta' \leq 0$,和第二种情况类似,我们有:

\[
\begin{array}{rll}
\Delta H & = max(|R'|, |L'|) - max (|R|, |L|) & \{\Delta \geq 0 \land \Delta' \leq 0 \}\\
         & = |L'|-|R| & \\
         & = |L|+\Delta H_l - |R| & \\
         & = \Delta H_l - \Delta&
\end{array}
\]

\item 最后一种情况,$\Delta$和$\Delta'$都不大于0,说明插入前后左子树的高度都不小于右子树。所以高度的增加全部“贡献”自左子树的变化$\Delta H_l$。
\end{itemize}

The next problem in front of us is how to determine the new balancing
factor value $\Delta'$ before performing balancing adjustment.
According to the definition of AVL tree, the balancing factor is the
height of right sub tree minus the height of left sub tree. We have
the following facts.

\be
\begin{array}{rl}
\Delta' & = |R'| - |L'| \\
        & = |R| + \Delta H_r - (|L| + \Delta H_l) \\
        & = |R| - |L| + \Delta H_r - \Delta H_l \\
        & = \Delta + \Delta H_r - \Delta H_l
\end{array}
\ee

With all these changes in height and balancing factor get clear, it's
possible to define the $tree()$ function mentioned in (\ref{eq:ins}).

\be
tree((L', \Delta H_l), Key, (R', \Delta H_r), \Delta) =
  balance (node(L', Key, R', \Delta'), \Delta H)
\ee

Before we moving into details of balancing adjustment, let's translate
the above equations to real programs in Haskell.

First is the insert function.

\lstset{language=Haskell}
\begin{lstlisting}
insert::(Ord a)=>AVLTree a -> a -> AVLTree a
insert t x = fst $ ins t where
    ins Empty = (Br Empty x Empty 0, 1)
    ins (Br l k r d)
        | x < k     = tree (ins l) k (r, 0) d
        | x == k    = (Br l k r d, 0)
        | otherwise = tree (l, 0) k (ins r) d
\end{lstlisting} %$

Here we also handle the case that inserting a duplicated key (which
means the key has already existed.) as just overwriting.

\begin{lstlisting}
tree::(AVLTree a, Int) -> a -> (AVLTree a, Int) -> Int -> (AVLTree a, Int)
tree (l, dl) k (r, dr) d = balance (Br l k r d', delta) where
    d' = d + dr - dl
    delta = deltaH d d' dl dr
\end{lstlisting}

And the definition of height increment is as below.

\begin{lstlisting}
deltaH :: Int -> Int -> Int -> Int -> Int
deltaH d d' dl dr
       | d >=0 && d' >=0 = dr
       | d <=0 && d' >=0 = d+dr
       | d >=0 && d' <=0 = dl - d
       | otherwise = dl
\end{lstlisting}

\subsection{Balancing adjustment}
\index{AVL tree!balancing}
As the pattern matching approach is adopted in doing re-balancing.
We need consider what kind of patterns violate the AVL tree property.

Figure \ref{fig:insert-fix} shows the 4 cases which need fix. For all
these 4 cases the balancing factors are either -2, or +2 which exceed
the range of $[-1, 1]$. After balancing adjustment, this factor turns
to be 0, which means the height of left sub tree is equal to the right
sub tree.

\begin{figure}[htbp]
   \begin{center}
     \setlength{\unitlength}{1cm}
     \begin{picture}(15, 15)
        % arrows
        \put(4.5, 9.5){\vector(1, -1){1}}
        \put(4.5, 5){\vector(1, 1){1}}
        \put(10, 9.5){\vector(-1, -1){1}}
        \put(10, 5){\vector(-1, 1){1}}
        % delta values
        \put(5, 13){$\delta(z) = -2$}
        \put(2.5, 12){$\delta(y) = -1$}
        \put(10, 13){$\delta(x) = 2$}
        \put(11.5, 11.5){$\delta(y) = 1$}
        \put(1.5, 5.5){$\delta(z) = -2$}
        \put(3.5, 4){$\delta(x) = 1$}
        \put(12, 5.5){$\delta(x) = 2$}
        \put(10.5, 4){$\delta(z) = -1$}
        \put(7.5, 10){$\delta'(y) = 0$}
        % graphics
	\put(0, 7){\includegraphics[scale=0.5]{img/insert-ll.ps}}
        \put(0, 0){\includegraphics[scale=0.5]{img/insert-lr.ps}}
        \put(7, 7){\includegraphics[scale=0.5]{img/insert-rr.ps}}
        \put(8.5, 0){\includegraphics[scale=0.5]{img/insert-rl.ps}}
        \put(2, 5){\includegraphics[scale=0.5]{img/insert-fixed.ps}}
      \end{picture}
     \caption{4 cases for balancing a AVL tree after insertion} \label{fig:insert-fix}
  \end{center}
\end{figure}

We call these four cases left-left lean, right-right lean, right-left lean,
and left-right lean cases in clock-wise direction from top-left. We denote
the balancing factor before fixing as $\delta(x), \delta(y)$, and $\delta(z)$, while after fixing, they changes to $\delta'(x), \delta'(y)$, and
$\delta'(z)$ respectively.

We'll next prove that, after fixing, we have $\delta(y)=0$ for all
four cases, and we'll provide the result values of $\delta'(x)$ and
$\delta'(z)$.

\subsubsection*{Left-left lean case}

As the structure of sub tree $x$ doesn't change due to fixing, we immediately get
$\delta'(x) = \delta(x)$.

Since $\delta(y) = -1$ and $\delta(z) = -2$, we have

\be
  \begin{array}{l}
  \delta(y) = |C| - |x| = -1 \Rightarrow |C| = |x| - 1 \\
  \delta(z) = |D| - |y| = -2 \Rightarrow |D| = |y| - 2
  \end{array}
  \label{eq:ll-cd}
\ee

After fixing.

\be
  \begin{array}{rll}
  \delta'(z) & = |D| - |C| & \{ From (\ref{eq:ll-cd}) \}\\
             & = |y| - 2 - (|x| - 1) & \\
             & = |y| - |x| - 1 & \{  x \text{ is child of } y \Rightarrow |y|-|x| = 1\} \\
             & = 0 &
  \end{array}
  \label{eq:ll-delta-z}
\ee

For $\delta'(y)$, we have the following fact after fixing.

\be
  \begin{array}{rll}
  \delta'(y) & = |z| - |x| & \\
             & = 1 + max(|C|, |D|) - |x| & \{ \text{By (\ref{eq:ll-delta-z}), we have} |C| = |D|\} \\
             & = 1 + |C| - |x| & \{ \text{By (\ref{eq:ll-cd})}\} \\
             & = 1 + |x| - 1 - |x| & \\
             & = 0 &
  \end{array}
\ee

Summarize the above results, the left-left lean case adjust the balancing
factors as the following.

\be
  \begin{array}{l}
  \delta'(x) = \delta(x) \\
  \delta'(y) = 0 \\
  \delta'(z) = 0
  \end{array}
\ee

\subsubsection*{Right-right lean case}

Since right-right case is symmetric to left-left case, we can easily achieve the result balancing factors as

\be
  \begin{array}{l}
  \delta'(x) = 0 \\
  \delta'(y) = 0 \\
  \delta'(z) = \delta(z)
  \end{array}
  \label{eq:rr-result}
\ee

\subsubsection*{Right-left lean case}

First let's consider $\delta'(x)$. After balance fixing, we have.

\be
  \delta'(x) = |B| - |A|
  \label{eq:rl-dx}
\ee

Before fixing, if we calculate the height of $z$, we can get.

\be
  \begin{array}{rll}
  |z| & = 1 + max(|y|, |D|) &  \{ \delta(z) = -1 \Rightarrow |y| > |D|\} \\
      & = 1 + |y| & \\
      & = 2 + max(|B|, |C|)
  \end{array}
  \label{eq:rl-z}
\ee

While since $\delta(x) = 2$, we can deduce that.

\be
  \begin{array}{rll}
  \delta(x) = 2 & \Rightarrow |z| - |A| = 2 & \{ \text{By (\ref{eq:rl-z})} \}\\
                & \Rightarrow 2 + max(|B|, |C|) - |A| = 2 & \\
                & \Rightarrow max(|B|, |C|) - |A| = 0 &
  \end{array}
  \label{eq:rl-ca}
\ee

If $\delta(y) = 1$, which means $|C| - |B| = 1$, it means

\be
  max(|B|, |C|)= |C| = |B|+1
\ee

Take this into (\ref{eq:rl-ca}) yields

\be
  \begin{array}{ll}
  |B|+1-|A| = 0 \Rightarrow |B|-|A|= -1 & \{ \text{By (\ref{eq:rl-dx}) } \} \\
  \Rightarrow \delta'(x) = -1 &
  \end{array}
\ee

If $\delta(y) \neq 1$, it means $max(|B|, |C|) = |B|$, taking this into
(\ref{eq:rl-ca}), yields.

\be
  \begin{array}{ll}
  |B| - |A| = 0  & \{ \text{By (\ref{eq:rl-dx})} \} \\
  \Rightarrow \delta'(x) = 0 &
  \end{array}
\ee

Summarize these 2 cases, we get relationship of $\delta'(x)$ and
$\delta(y)$ as the following.

\be
\delta'(x) = \left \{
  \begin{array}
  {r@{\quad:\quad}l}
  -1 & \delta(y) = 1 \\
  0 & otherwise
  \end{array}
\right.
\label{eq:rl-dx-dy}
\ee

For $\delta'(z)$ according to definition, it is equal to.

\be
  \begin{array}{rll}
    \delta'(z) & = |D| - |C| & \{ \delta(z) = -1 = |D| - |y| \} \\
               & = |y| - |C| - 1 & \{ |y| = 1 + max(|B|, |C|) \} \\
               & = max(|B|, |C|) - |C|
  \end{array}
  \label{eq:rl-dz}
\ee

If $\delta(y) = -1$, then we have $|C| - |B| = -1$, so $max(|B|, |C|) = |B| = |C| + 1$. Takes this into (\ref{eq:rl-dz}), we get $\delta'(z) = 1$.

If $\delta(y) \neq -1$, then $max(|B|, |C|) = |C|$, we get $\delta'(z)=0$.

Combined these two cases, the relationship between $\delta'(z)$ and $\delta(y)$ is as below.

\be
\delta'(z) = \left \{
  \begin{array}
  {r@{\quad:\quad}l}
  1 & \delta(y) = -1 \\
  0 & otherwise
  \end{array}
  \right.
  \label{eq:rl-dz-dy}
\ee

Finally, for $\delta'(y)$, we deduce it like below.

\be
  \begin{array}{rl}
  \delta'(y) & = |z| - |x| \\
             & = max(|C|, |D|) - max(|A|, |B|)
  \end{array}
  \label{eq:rl-dy}
\ee

There are three cases.
\begin{itemize}

\item If $\delta(y)=0$, it means $|B|=|C|$, and according to (\ref{eq:rl-dx-dy}) and (\ref{eq:rl-dz-dy}), we have $\delta'(x)=0 \Rightarrow |A| = |B|$, and $\delta'(z)=0 \Rightarrow |C|=|D|$, these lead to $\delta'(y)=0$.

\item If $\delta(y)=1$, From (\ref{eq:rl-dz-dy}), we have $\delta'(z)=0 \Rightarrow |C| = |D|$.
\[
  \begin{array}{rll}
  \delta'(y) & = max(|C|, |D|) - max(|A|, |B|) & \{|C|=|D|\} \\
             & = |C| - max(|A|, |B|) & \{\text{From (\ref{eq:rl-dx-dy}): $\delta'(x)=-1 \Rightarrow |B|-|A|=-1$} \} \\
             & = |C| - (|B| + 1) & \{ \delta(y) = 1 \Rightarrow |C|-|B|=1\} \\
             & = 0
  \end{array}
\]

\item If $\delta(y)=-1$, From (\ref{eq:rl-dx-dy}), we have $\delta'(x)=0 \Rightarrow |A|=|B|$.
\[
  \begin{array}{rll}
  \delta'(y) & = max(|C|, |D|) - max(|A|, |B|) & \{|A|=|B|\} \\
             & = max(|C|, |D|) - |B| & \{ \text{From (\ref{eq:rl-dz-dy}): $|D|-|C|=1$} \} \\
             & = |C| + 1 - |B| & \{  \delta(y) = -1 \Rightarrow |C|-|B|=-1\} \\
             & = 0
  \end{array}
\]

\end{itemize}

Both three cases lead to the same result that $\delta'(y)=0$.

Collect all the above results, we get the new balancing factors after fixing as the following.

\be
  \begin{array}{l}
  \delta'(x) = \left \{
    \begin{array}
    {r@{\quad:\quad}l}
    -1 & \delta(y) = 1 \\
    0 & otherwise
    \end{array}
    \right. \\
  \delta'(y) = 0 \\
  \delta'(z) = \left \{
    \begin{array}
    {r@{\quad:\quad}l}
    1 & \delta(y) = -1 \\
    0 & otherwise
    \end{array}
    \right.
  \end{array}
  \label{eq:rl-result}
\ee

\subsubsection*{Left-right lean case}

Left-right lean case is symmetric to the Right-left lean case. By using
the similar deduction, we can find the new balancing factors are identical
to the result in (\ref{eq:rl-result}).

\subsection{Pattern Matching}
All the problems have been solved and it's time to define the final
pattern matching fixing function.

\be
balance(T, \Delta H) = \left \{
  \begin{array}
  {r@{\quad:\quad}l}
  (node(node(A, x, B, \delta(x)), y, node(C, z, D, 0), 0), 0) & P_{ll}(T) \\
  (node(node(A, x, B, 0), y, node(C, z, D, \delta(z)), 0), 0) & P_{rr}(T) \\
  (node(node(A, x, B, \delta'(x)), y, node(C, z, D, \delta'(z)), 0), 0) & P_{rl}(T) \lor P_{lr}(T) \\
  (T, \Delta H) & otherwise
  \end{array}
\right.
\ee

Where $P_{ll}(T)$ means the pattern of tree $T$ is left-left lean respectively. $\delta'(x)$ and $delta'(z)$ are defined in (\ref{eq:rl-result}). The four patterns are tested as below.

\be
\begin{array}{l}
P_{ll}(T) = node(node(node(A, x, B, \delta(x)), y, C, -1), z, D, -2) \\
P_{rr}(T) = node(A, x, node(B, y, node(C, z, D, \delta(z)), 1), 2) \\
P_{rl}(T) = node(node(A, x, node(B, y, C, \delta(y)), 1), z, D, -2) \\
P_{lr}(T) = node(A, x, node(node(B, y, C, \delta(y)), z, D, -1), 2)
\end{array}
\ee

Translating the above function definition to Haskell yields a simple
and intuitive program.

\begin{lstlisting}
balance :: (AVLTree a, Int) -> (AVLTree a, Int)
balance (Br (Br (Br a x b dx) y c (-1)) z d (-2), _) =
        (Br (Br a x b dx) y (Br c z d 0) 0, 0)
balance (Br a x (Br b y (Br c z d dz)    1)    2, _) =
        (Br (Br a x b 0) y (Br c z d dz) 0, 0)
balance (Br (Br a x (Br b y c dy)    1) z d (-2), _) =
        (Br (Br a x b dx') y (Br c z d dz') 0, 0) where
    dx' = if dy ==  1 then -1 else 0
    dz' = if dy == -1 then  1 else 0
balance (Br a x (Br (Br b y c dy) z d (-1))    2, _) =
        (Br (Br a x b dx') y (Br c z d dz') 0, 0) where
    dx' = if dy ==  1 then -1 else 0
    dz' = if dy == -1 then  1 else 0
balance (t, d) = (t, d)
\end{lstlisting}

The insertion algorithm takes time proportion to the height of the
tree, and according to the result we proved above, its performance
is $O(\lg N)$ where $N$ is the number of elements stored in the AVL
tree.

\subsubsection{Verification}
\index{AVL tree!verification}
One can easily create a function to verify a tree is AVL tree.
Actually we need verify two things, first, it's a binary search tree;
second, it satisfies AVL tree property.

We left the first verification problem as an exercise to the reader.

In order to test if a binary tree satisfies AVL tree property, we can
test the difference in height between its two children, and recursively
test that both children conform to AVL property until we arrive at
an empty leaf.

\be
  avl?(T) = \left \{
  \begin{array}
  {r@{\quad:\quad}l}
  True & T = \Phi \\
  avl?(L) \land avl?(R) \land ||R|-|L|| \leq 1 & otherwise
  \end{array}
  \right .
\ee

And the height of a AVL tree can also be calculate from the definition.

\be
  |T| = \left \{
  \begin{array}
  {r@{\quad:\quad}l}
  0 & T = \Phi \\
  1 + max(|R|, |L|) & otherwise
  \end{array}
  \right .
\ee

The corresponding Haskell program is given as the following.

\begin{lstlisting}
isAVL :: (AVLTree a) -> Bool
isAVL Empty = True
isAVL (Br l _ r d) = and [isAVL l, isAVL r, abs (height r - height l) <= 1]

height :: (AVLTree a) -> Int
height Empty = 0
height (Br l _ r _) = 1 + max (height l) (height r)
\end{lstlisting}

\begin{Exercise}
Write a program to verify a binary tree is a binary search tree in your
favorite programming language. If you choose to use an imperative language,
please consider realize this program without recursion.
\end{Exercise}



% ================================================================
%                 Deletion
% ================================================================

\section{Deletion}
\index{AVL tree!deletion}

As we mentioned before, deletion doesn't make significant sense in
purely functional settings. As the tree is read only, it's typically
performs frequently looking up after build.

Even if we implement deletion, it's actually re-building the tree
as we presented in chapter of red-black tree. We left the deletion
of AVL tree as an exercise to the reader.

\begin{Exercise}

\begin{itemize}

\item Take red-black tree deletion algorithm as an example, write the
AVL tree deletion program in purely functional approach in your
favorite programming language.

\item Write the deletion algorithm in imperative approach in your favorite
programming language.

\end{itemize}

\end{Exercise}

\section{Imperative AVL tree algorithm $\star$}
\index{AVL tree!imperative insertion}

We almost finished all the content in this chapter about AVL tree.
However, it necessary to show the traditional insert-and-rotate
approach as the comparator to pattern matching algorithm.

Similar as the imperative red-black tree algorithm, the strategy
is first to do the insertion as same as for binary search tree,
then fix the balance problem by rotation and return the final result.

\begin{algorithmic}[1]
\Function{Insert}{$T, k$}
  \State $root \gets T$
  \State $x \gets$ \Call{Create-Leaf}{$k$}
  \State \Call{$\delta$}{$x$} $\gets 0$
  \State $parent \gets NIL$
  \While{$T \neq NIL$}
    \State $parent \gets T$
    \If{$k <$ \Call{Key}{$T$}}
      \State $T \gets $ \Call{Left}{$T$}
    \Else
      \State $T \gets $ \Call{Right}{$T$}
    \EndIf
  \EndWhile
  \State \Call{Parent}{$x$} $\gets parent$
  \If{$parent = NIL$} \Comment{tree $T$ is empty}
    \State \Return $x$
  \ElsIf{$k <$ \Call{Key}{$parent$}}
    \State \Call{Left}{$parent$} $\gets x$
  \Else
    \State \Call{Right}{$parent$} $\gets x$
  \EndIf
  \State \Return \Call{AVL-Insert-Fix}{$root, x$}
\EndFunction
\end{algorithmic}

Note that after insertion, the height of the tree may increase, so that
the balancing factor $\delta$ may also change, insert on right side will
increase $\delta$ by 1, while insert on left side will decrease it. By
the end of this algorithm, we need perform bottom-up fixing from node $x$
towards root.

We can translate the pseudo code to real programming language, such as
Python \footnote{C and C++ source code are available along with this book}.
\lstset{language=Python}
\begin{lstlisting}
def avl_insert(t, key):
    root = t
    x = Node(key)
    parent = None
    while(t):
        parent = t
        if(key < t.key):
            t = t.left
        else:
            t = t.right
    if parent is None: #tree is empty
        root = x
    elif key < parent.key:
        parent.set_left(x)
    else:
        parent.set_right(x)
    return avl_insert_fix(root, x)
\end{lstlisting}

This is a top-down algorithm search the tree from root down to the proper
position and insert the new key as a leaf. By the end of this algorithm, it calls fixing procedure, by passing the root and the new node inserted.

Note that we reuse the same methods of set\_left() and set\_right() as
we defined in chapter of red-black tree.

In order to resume the AVL tree balance property by fixing, we first determine if the new node is inserted on left hand or right hand. If it is on left, the balancing factor $\delta$ decreases, otherwise it increases. If we denote the new value as $\delta'$, there are 3 cases of the relationship between $\delta$ and $\delta'$.

\begin{itemize}
\item If $|\delta| = 1$ and $|\delta'| = 0$, this means adding the new node makes the tree perfectly balanced, the height of the parent node doesn't change, the algorithm can be terminated.

\item If $|\delta| = 0$ and $|\delta'| = 1$, it means that either the height of left sub tree or right sub tree increases, we need go on check the upper level of the tree.

\item If $|\delta| = 1$ and $|\delta'| = 2$, it means the AVL tree property is violated due to the new insertion. We need perform rotation to fix it.
\end{itemize}

\begin{algorithmic}[1]
\Function{AVL-Insert-Fix}{$T, x$}
  \While{\Call{Parent}{$x$} $\neq NIL$}
    \State $\delta \gets $ \textproc{$\delta$}(\Call{Parent}{$x$})
    \If{$x = $ \textproc{Left}(\Call{Parent}{$x$})}
      \State $\delta' \gets \delta - 1$
    \Else
      \State $\delta' \gets \delta + 1$
    \EndIf
    \State \textproc{$\delta$}(\Call{Parent}{$x$}) $\gets \delta'$
    \State $P \gets $ \Call{Parent}{$x$}
    \State $L \gets $ \Call{Left}{$x$}
    \State $R \gets $ \Call{Right}{$x$}
    \If{$|\delta| = 1$ and $|\delta'| = 0$} \Comment{Height doesn't change, terminates.}
      \State \Return $T$
    \ElsIf{$|\delta| = 0$ and $|\delta'| = 1$} \Comment{Go on bottom-up updating.}
      \State $x \gets P$
    \ElsIf{$|\delta| = 1$ and $|\delta'| = 2$}
      \If{$\delta'=2$}
        \If{$\delta(R) = 1$} \Comment{Right-right case}
          \State $\delta(P) \gets 0$ \Comment{By (\ref{eq:rr-result})}
          \State $\delta(R) \gets 0$
          \State $T \gets $ \Call{Left-Rotate}{$T, P$}
        \EndIf
        \If{$\delta(R) = -1$} \Comment{Right-left case}
          \State $\delta_y \gets $ \textproc{$\delta$}(\Call{Left}{$R$}) \Comment{By (\ref{eq:rl-result})}
          \If{$\delta_y = 1$}
            \State $\delta(P) \gets -1$
          \Else
            \State $\delta(P) \gets 0$
          \EndIf
          \State \textproc{$\delta$}(\Call{Left}{$R$}) $\gets 0$
          \If{$\delta_y = -1$}
            \State $\delta(R) \gets 1$
          \Else
            \State $\delta(R) \gets 0$
          \EndIf
          \State $T \gets $ \Call{Right-Rotate}{$T, R$}
          \State $T \gets $ \Call{Left-Rotate}{$T, P$}
        \EndIf
      \EndIf
      \If{$\delta' = -2$}
        \If{$\delta(L) = -1$} \Comment{Left-left case}
          \State $\delta(P) \gets 0$
          \State $\delta(L) \gets 0$
          \State \Call{Right-Rotate}{$T, P$}
        \Else \Comment{Left-Right case}
          \State $\delta_y \gets $ \textproc{$\delta$}(\Call{Right}{$L$})
          \If{$\delta_y = 1$}
            \State $\delta(L) \gets -1$
          \Else
            \State $\delta(L) \gets 0$
          \EndIf
          \State \textproc{$\delta$}(\Call{Right}{$L$}) $\gets 0$
          \If{$\delta_y = -1$}
            \State $\delta(P) \gets 1$
          \Else
            \State $\delta(P) \gets 0$
          \EndIf
          \State \Call{Left-Rotate}{$T, L$}
          \State \Call{Right-Rotate}{$T, P$}
        \EndIf
      \EndIf
      \State break
    \EndIf
  \EndWhile
  \State \Return $T$
\EndFunction
\end{algorithmic}

Here we reuse the rotation algorithms mentioned in red-black tree chapter.
Rotation operation doesn't update balancing factor $\delta$ at all,
However, since rotation changes (actually improves) the balance situation
we should update these factors. Here we refer the results from above section. Among the four cases, right-right case and left-left case only need one rotation, while right-left case and left-right case need two rotations.

The relative python program is shown as the following.

\begin{lstlisting}
def avl_insert_fix(t, x):
    while x.parent is not None:
        d2 = d1 = x.parent.delta
        if x == x.parent.left:
            d2 = d2 - 1
        else:
            d2 = d2 + 1
        x.parent.delta = d2
        (p, l, r) = (x.parent, x.parent.left, x.parent.right)
        if abs(d1) == 1 and abs(d2) == 0:
            return t
        elif abs(d1) == 0 and abs(d2) == 1:
            x = x.parent
        elif abs(d1)==1 and abs(d2) == 2:
            if d2 == 2:
                if r.delta == 1:  # Right-right case
                    p.delta = 0
                    r.delta = 0
                    t = left_rotate(t, p)
                if r.delta == -1: # Right-Left case
                    dy = r.left.delta
                    if dy == 1:
                        p.delta = -1
                    else:
                        p.delta = 0
                    r.left.delta = 0
                    if dy == -1:
                        r.delta = 1
                    else:
                        r.delta = 0
                    t = right_rotate(t, r)
                    t = left_rotate(t, p)
            if d2 == -2:
                if l.delta == -1: # Left-left case
                    p.delta = 0
                    l.delta = 0
                    t = right_rotate(t, p)
                if l.delta == 1: # Left-right case
                    dy = l.right.delta
                    if dy == 1:
                        l.delta = -1
                    else:
                        l.delta = 0
                    l.right.delta = 0
                    if dy == -1:
                        p.delta = 1
                    else:
                        p.delta = 0
                    t = left_rotate(t, l)
                    t = right_rotate(t, p)
            break
    return t
\end{lstlisting}

We skip the AVL tree deletion algorithm and left this as an exercise to the reader.

\section{Chapter note}
AVL tree was invented in 1962 by Adelson-Velskii and Landis\cite{wiki},
\cite{TFATP}. The name AVL tree comes from the two inventor's name. It's earlier than red-black tree.

It's very common to compare AVL tree and red-black tree, both are self-balancing binary search trees, and for all the major operations, they both consume $O(\lg N)$ time. From the result of (\ref{eq:AVL-height}), AVL tree is more rigidly balanced hence they are faster than red-black tree in looking up intensive applications \cite{wiki}. However, red-black trees could perform better in frequently insertion and removal cases.

Many popular self-balancing binary search tree libraries are implemented on top of red-black tree such as STL etc. However, AVL tree provides an intuitive and effective solution to the balance problem as well.

After this chapter, we'll extend the tree data structure from storing data in node to storing information on edges, which leads to Trie and Patrica, etc. If we extend the number of children from two to more, we can get B-tree. These data structures will be introduced next.

\begin{thebibliography}{99}

\bibitem{hackage}
Data.Tree.AVL http://hackage.haskell.org/packages/archive/AvlTree/4.2/doc/html/Data-Tree-AVL.html

\bibitem{okasaki}
Chris Okasaki. ``FUNCTIONAL PEARLS Red-Black Trees in a Functional Setting''. J. Functional Programming. 1998

\bibitem{wiki}
Wikipedia. ``AVL tree''. http://en.wikipedia.org/wiki/AVL\_tree

\bibitem{TFATP}
Guy Cousinear, Michel Mauny. ``The Functional Approach to Programming''. Cambridge University Press; English Ed edition (October 29, 1998). ISBN-13: 978-0521576819

\bibitem{py-avl}
Pavel Grafov. ``Implementation of an AVL tree in Python''. http://github.com/pgrafov/python-avl-tree
\end{thebibliography}

\ifx\wholebook\relax\else
\end{document}
\fi

% LocalWords:  AVL Okasaki STL
