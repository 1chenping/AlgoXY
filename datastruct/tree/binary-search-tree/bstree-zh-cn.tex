\ifx\wholebook\relax \else
% ------------------------

\documentclass[UTF8]{article}
%
% loading packages
%

\RequirePackage{ifpdf}
\RequirePackage{ifxetex}

%
%
\ifpdf
  \RequirePackage[pdftex,%
       bookmarksnumbered,%
              colorlinks,%
          linkcolor=blue,%
              hyperindex,%
        plainpages=false,%
       pdfstartview=FitH]{hyperref}
\else\ifxetex
  \RequirePackage[bookmarksnumbered,%
               colorlinks,%
           linkcolor=blue,%
               hyperindex,%
         plainpages=false,%
        pdfstartview=FitH]{hyperref}
\else
  \RequirePackage[dvipdfm,%
        bookmarksnumbered,%
               colorlinks,%
           linkcolor=blue,%
               hyperindex,%
         plainpages=false,%
        pdfstartview=FitH]{hyperref}
\fi\fi
%\usepackage{hyperref}

% other packages
%--------------------------------------------------------------------------
\usepackage{graphicx, color}
\usepackage{subfig}
\usepackage{tikz}
\usetikzlibrary{matrix,positioning}

\usepackage{amsmath, amsthm, amssymb} % for math
\usepackage{exercise} % for exercise
\usepackage{import} % for nested input

%
% for programming
%
\usepackage{verbatim}
\usepackage{listings}
%\usepackage{algorithmic} %old version; we can use algorithmicx instead
\usepackage{algorithm}
\usepackage[noend]{algpseudocode} %for pseudo code, include algorithmicsx automatically
\usepackage{appendix}
\usepackage{makeidx} % for index support
\usepackage{titlesec}

\usepackage[cm-default]{fontspec}
\usepackage{xunicode}

% detect and select Chinese font
% ------------------------------
% the following cmd can list all availabe Chinese fonts in host.
% fc-list :lang=zh
\def\myfont{STHeiti}  % Under Mac OS X
\def\linuxfallback{WenQuanYi Micro Hei} % Under Linux
\def\winfallback{SimSun} % Under Windows
\suppressfontnotfounderror1 % Avoid setting exit code (error level) to break make process
\count255=\interactionmode
\batchmode
\font\foo="\myfont"\space at 10pt
\ifx\foo\nullfont
  \font\foo = "\linuxfallback"\space at 10pt
  \ifx\foo\nullfont
    \font\foo = "\winfallback"\space at 10pt
    \ifx\foo\nullfont
      \errorstopmode
      \errmessage{no suitable Chinese font found}
    \else
      \let\myfont=\winfallback % Windows
    \fi
  \else
    \let\myfont=\linuxfallback % Linux
  \fi
\fi
\interactionmode=\count255
\setmainfont[Mapping=tex-text]{\myfont}

\XeTeXlinebreaklocale "zh"  % to solve the line breaking issue
\XeTeXlinebreakskip = 0pt plus 1pt minus 0.1pt

\titleformat{\paragraph}
{\normalfont\normalsize\bfseries}{\theparagraph}{1em}{}
\titlespacing*{\paragraph}
{0pt}{3.25ex plus 1ex minus .2ex}{1.5ex plus .2ex}

\lstdefinelanguage{Smalltalk}{
  morekeywords={self,super,true,false,nil,thisContext}, % This is overkill
  morestring=[d]',
  morecomment=[s]{"}{"},
  alsoletter={\#:},
  escapechar={!},
  literate=
    {BANG}{!}1
    {UNDERSCORE}{\_}1
    {\\st}{Smalltalk}9 % convenience -- in case \st occurs in code
    % {'}{{\textquotesingle}}1 % replaced by upquote=true in \lstset
    {_}{{$\leftarrow$}}1
    {>>>}{{\sep}}1
    {^}{{$\uparrow$}}1
    {~}{{$\sim$}}1
    {-}{{\sf -\hspace{-0.13em}-}}1  % the goal is to make - the same width as +
    %{+}{\raisebox{0.08ex}{+}}1		% and to raise + off the baseline to match -
    {-->}{{\quad$\longrightarrow$\quad}}3
	, % Don't forget the comma at the end!
  tabsize=2
}[keywords,comments,strings]

% for better Haskell code outlook
\lstdefinelanguage{Haskell}{
  basicstyle=\small\ttfamily,
  flexiblecolumns=false,
  basewidth={0.5em,0.45em},
  literate={+}{{$+$}}1 {/}{{$/$}}1 {*}{{$*$}}1 {=}{{$=$}}1
           {>}{{$>$}}1 {<}{{$<$}}1 {\\}{{$\lambda$}}1
           {\\\\}{{\char`\\\char`\\}}1
           {->}{{$\rightarrow$}}2 {>=}{{$\geq$}}2 {<-}{{$\leftarrow$}}2
           {<=}{{$\leq$}}2 {=>}{{$\Rightarrow$}}2
           {\ .}{{$\circ$}}2 {\ .\ }{{$\circ$}}2
           {>>}{{>>}}2 {>>=}{{>>=}}2
           {|}{{$\mid$}}1
}[keywords,comments,strings]

\lstloadlanguages{C, C++, Lisp, Haskell, Python, Smalltalk}

\lstset{
  showstringspaces = false
}

% ======================================================================

\def\BibTeX{{\rm B\kern-.05em{\sc i\kern-.025em b}\kern-.08em
    T\kern-.1667em\lower.7ex\hbox{E}\kern-.125emX}}

%
% mathematics
%
\newcommand{\be}{\begin{equation}}
\newcommand{\ee}{\end{equation}}
\newcommand{\bmat}[1]{\left( \begin{array}{#1} }
\newcommand{\emat}{\end{array} \right) }
\newcommand{\VEC}[1]{\mbox{\boldmath $#1$}}

% numbered equation array
\newcommand{\bea}{\begin{eqnarray}}
\newcommand{\eea}{\end{eqnarray}}

% equation array not numbered
\newcommand{\bean}{\begin{eqnarray*}}
\newcommand{\eean}{\end{eqnarray*}}

\newtheorem{theorem}{Theorem}[section]
\newtheorem{lemma}[theorem]{Lemma}
\newtheorem{proposition}[theorem]{Proposition}
\newtheorem{corollary}[theorem]{Corollary}


\setcounter{page}{1}

\begin{document}

%--------------------------

% ================================================================
%                 COVER PAGE
% ================================================================

\title{二叉搜索树,数据结构中的`hello world'}

\author{刘新宇
\thanks{{\bfseries 刘新宇} \newline
  Email: liuxinyu95@gmail.com \newline}
  }

\maketitle
\fi

\markboth{二叉搜索树}{初等算法}

\ifx\wholebook\relax
\chapter{二叉搜索树,数据结构中的`hello world'}
\numberwithin{Exercise}{chapter}
\fi

% ================================================================
%                 Introduction
% ================================================================
\section{简介}
\label{introduction} \index{二叉搜索树}

数组和链表通常被认为是最简单的“hello world”式数据结构。其实它们并不简单。在某些系统中,数组是最基本的数据结构,甚至链表也可以由数组来实现(《算法导论》第10.3节\cite{CLRS})。另一方面,在某些函数式环境中,链表被作为最基本的数据结构来实现数组和其他更复杂的数据结构。

考虑这些因素,我们使用二叉搜索树(BST)作为“hello world”数据结构。Jon Bentley在他的《编程珠玑》一书中,曾经给了这样一个有趣的题目\cite{Bentley}:如何统计一段文字中每个单词出现多少次?下面的C++程序展示了一个解法。\index{word counter}

\lstset{language=C++}
\begin{lstlisting}
int main(int, char** ){
  map<string, int> dict;
  string s;
  while(cin>>s)
    ++dict[s];
  map<string, int>::iterator it=dict.begin();
  for(; it!=dict.end(); ++it)
    cout<<it->first<<": "<<it->second<<"\n";
}
\end{lstlisting}

我们可以运行下面的UNIX命令获得对单词的统计结果。\footnote{在Windows系统中,相应的用法为:\texttt{type bbe.txt | wordcount.exe > wc.txt}}

\begin{verbatim}
$ g++ wordcount.cpp -o wordcount
$ cat bbe.txt | ./wordcount > wc.txt
\end{verbatim}

C++标准库中提供的\texttt{map}是一种用平衡二叉树实现的字典数据结构。例子中用单词作为key,用单词出现的次数作为值。这个例子程序运行快速,展示了二叉搜索树的强大。本章我们介绍二叉搜索树的实现,然后再后面的章节中介绍如何实现二叉树的平衡。

在我们正式开始前,先了解一下广义二叉树的定义。二叉搜索树只不过是一种特殊的二叉树,我们可以递归地定义二叉树如下\index{binary tree}:

一个二叉树
\begin{itemize}
\item 或者为空,
\item 或者包含三个部份:一个值,一个左侧分支和一个右侧分支,其中两个分支也都是二叉树。
\end{itemize}

左右分支也被称为左子树和右子树,或统称为孩子。一个树也被称为一个节点。节点中的值可以是任何类型,甚至为空。如果一个树的左右子树都为空,我们称之为叶子节点,否则称为分支节点。图\ref{fig:binary-tree-example}展示了二叉树的概念和例子。

\begin{figure}[htbp]
  \centering
  \subfloat[二叉树的概念]{\includegraphics[scale=0.5]{img/lvr.ps}} \\
  \subfloat[一个二叉树]{\includegraphics[scale=0.5]{img/btexample.ps}}
  \caption{二叉树的概念和例子。}
  \label{fig:binary-tree-example}
\end{figure}

一个二叉搜索树是一个满足下面条件的二叉树:
\begin{itemize}
\item 所有左侧分支的值都小于本节点的值,
\item 本节点的值小于所有右侧分支节点的值。
\end{itemize}

图\ref{fig:bst-example}展示了一个二叉搜索树的例子。和前面的图\ref{fig:binary-tree-example}比较,可以看到它们所包含值的组织方式是不同的。一个广义二叉树的值可以是任意类型,而二叉搜索树的定义要求它的值必须能比较大小\footnote{实际上只要能进行小于比较就足够了。}。为了强调这种区别,我们特别称二叉搜索树的的值为键(key),把节点存储的其他数据信息称为值(value)。

\begin{figure}[htbp]
  \centering
  \includegraphics[scale=0.5]{img/bst-1.ps}
  \caption{二叉搜索树的例子} \label{fig:bst-example}
\end{figure}


% ================================================================
% Data layout
% ================================================================
\section{数据组织}
\index{二叉搜索树!数据组织}

根据二叉搜索树的定义,在传统过程式编程环境中,我们可以用指针来描绘数据的组织结构。如图\ref{fig:node-layout-parent}。

%\begin{figure}[htbp]
%       \begin{center}
%        \includegraphics[scale=0.8]{img/node-layout.ps}
%        \caption{Node layout.} \label{fig:node-layout}
%       \end{center}
%\end{figure}

一个树的节点首先包含一个键,一个键可以附加一些额外的数据(也称为“satellite data”)。接下来分别是指向左右子树的两个指针。为了方便从一个节点上溯到祖先节点,有时也存储一个指向父亲的指针(称为“父指针”)。
%Figure \ref{fig:node-layout-parent} shows the layout with parent field.

\begin{figure}[htbp]
  \centering
  \includegraphics[scale=0.8]{img/node-layout-parent.ps}
  \caption{带有父指针的数据组织结构。} \label{fig:node-layout-parent}
\end{figure}

简单起见,本章中,我们暂时忽略“satellite data”。下面的C++例子代码依据上面的数据的组织方式定义了二叉搜索树的节点。

\lstset{language=C++}
\begin{lstlisting}
template<class T>
struct node{
  node(T x):key(x), left(0), right(0), parent(0){}
  ~node(){
    delete left;
    delete right;
  }

  node* left;
  node* right;
  node* parent; //Optional, it's helpful for succ and pred
  T key;
};
\end{lstlisting}

在以链表作为基本数据结构的环境中,例如Lisp,二叉搜索树也可以由链表来构建,如图\ref{fig:lisp-layout}。

\begin{figure}[htbp]
  \centering
  \includegraphics[scale=0.8]{img/lisp-layout.ps}
  \caption{由链表构件的二叉搜索树。其中\texttt{left...}和\texttt{right...}或者为空,或者是以同样方式构建的节点。}
  \label{fig:lisp-layout}
\end{figure}

在许多函数式环境中,难以用指针来进行回溯(通常以自顶向下的递归来代替回溯),所以数据组织上往往不使用“父节点”。

为了简化问题,我们在此后将跳过这样的具体数据组织细节,而只关注数据结构的逻辑。例如,下面的Haskell代码定义了二叉搜索树的节点。

\lstset{language=Haskell}
\begin{lstlisting}
data Tree a = Empty
            | Node (Tree a) a (Tree a)
\end{lstlisting}

% ================================================================
% Insert
% ================================================================
\section{插入}
\index{二叉搜索树!插入}

我们可以使用下述算法向向一个二叉搜索树中插入一个键$k$(在实际应用中,有时会同时插入一对键和值):

\begin{itemize}
\item 如果树为空,创建一个叶子节点,该节点的key = $k$,
\item 如果$k$小于根节点的key,将它插入到左子树中去,
\item 如果$k$大于根节点的key,将它插入到右子树中去。
\end{itemize}

这里存在一个特殊情况,当$k$等于根节点的key时,说明它已经存在了。我们可以覆盖(overwrite)掉以前的数据,也可以选择跳过不做任何处理。简单起见,我们忽略这一情况。

插入算法是递归的。它十分简单。因此我们说二叉搜索树是“hello world”数据结构。这一算法可以形式化地定义为如下的函数。

\be
insert(T, k) = \left \{
  \begin{array}
  {r@{\quad:\quad}l}
  node(\phi, k, \phi) & T = \phi \\
  node(insert(T_l, k), k', T_r) & k < k' \\
  node(T_l, k', insert(T_r, k)) & otherwise
  \end{array}
\right.
\ee

其中,当$T$不为空时,$T_l$, $T_r$和$k'$分别是它的左右子树和key。函数$node$以给定的左子树、key和右子树为参数创建一个新节点。符号$\phi$表示NIL或空。

将上述函数直接翻译为Haskell代码可以得到下面的程序:

\lstset{language=Haskell}
\begin{lstlisting}
insert::(Ord a) => Tree a -> a -> Tree a
insert Empty k = Node Empty k Empty
insert (Node l x r) k | k < x = Node (insert l k) x r
                      | otherwise = Node l x (insert r k)
\end{lstlisting}

这一程序使用了语言提供的pattern matching特性。但即使不用这一特性(例如Lisp方言Scheme),函数式的插入程序仍然十分简洁。

\lstset{language=lisp}
\begin{lstlisting}
(define (insert tree x)
  (cond ((null? tree) (list '() x '()))
	((< x (key tree))
	 (make-tree (insert (left tree) x)
		    (key tree)
		    (right tree)))
	((> x (key tree))
	 (make-tree (left tree)
		    (key tree)
		    (insert (right tree) x)))))
\end{lstlisting}

这一算法也可以完全不用递归,而用imperative的方式实现:

\begin{algorithmic}[1]
\Function{Insert}{$T, k$}
  \State $root \gets T$
  \State $x \gets$ \Call{Create-Leaf}{$k$}
  \State $parent \gets NIL$
  \While{$T \neq NIL$}
    \State $parent \gets T$
    \If{$k <$ \Call{Key}{$T$}}
      \State $T \gets $ \Call{Left}{$T$}
    \Else
      \State $T \gets $ \Call{Right}{$T$}
    \EndIf
  \EndWhile
  \State \Call{Parent}{$x$} $\gets parent$
  \If{$parent = NIL$} \Comment{tree $T$ is empty}
    \State \Return $x$
  \ElsIf{$k <$ \Call{Key}{$parent$}}
    \State \Call{Left}{$parent$} $\gets x$
  \Else
    \State \Call{Right}{$parent$} $\gets x$
  \EndIf
  \State \Return $root$
\EndFunction
\Statex
\Function{Create-Leaf}{k}
  \State $x \gets $ \Call{Empty-Node}{}
  \State \Call{Key}{$x$} $ \gets k$
  \State \Call{Left}{$x$} $ \gets NIL$
  \State \Call{Right}{$x$} $ \gets NIL$
  \State \Call{Parent}{$x$} $ \gets NIL$
  \State \Return $x$
\EndFunction
\end{algorithmic}

虽然这样没有函数式的简洁,但是它的速度很快,并且可以处理深度很大的树。限于篇幅,相应完整的C++和Python程序就不再列出了。读者可以从本书的网站下载参考。

\section{遍历}
\index{二叉树!遍历}

遍历是指依次访问二叉树中的每个元素。有三种遍历方法,分别是前序遍历、中序遍历和后序遍历。它们是按照访问根节点和子节点的先后顺序命名的。

\begin{itemize}
\item 前序遍历:先访问key,然后访问左子树,最后访问右子树;
\item 中序遍历:先访问左子树,然后访问key,最后访问右子树;
\item 后序遍历:先访问左子树,然后访问右子树,最后访问key。
\end{itemize}

\index{前序遍历} \index{中序遍历} \index{后序遍历}

所有的“访问”操作都是递归的。\underline{先}访问根后访问子分支称为\underline{先序},在访问左右分支的\underline{中间}访问根称为\underline{中序},先访问子分支\underline{后}访问根称为\underline{后序}。

对于图\ref{fig:bst-example}中的二叉树,下面分别列出了三种遍历的结果:

\begin{itemize}
\item 前序遍历:4, 3, 1, 2, 8, 7, 16, 10, 9, 14;
\item 中序遍历:1, 2, 3, 4, 7, 8, 9, 10, 14, 16;
\item 后序遍历:2, 1, 3, 7, 9, 14, 10, 16, 8, 4。
\end{itemize}

对二叉搜索树进行中序遍历,元素就会按照从小到大输出。二叉搜索树的定义保证了这一有趣的性质,作为练习,请读者思考如何证明。

中序遍历的算法可以描述为:
\begin{itemize}
\item 如果树为空,则返回;
\item 否则先中序遍历左子树,然后访问key,最后再中序遍历右子树。
\end{itemize}

这一描述本身是递归的,如果二叉树非空,令$T_l$, $T_r$和$k$分别代表左右子树和key。我们可以定义下面的抽象map函数。

\be
map(f, T) = \left \{
  \begin{array}
  {r@{\quad:\quad}l}
  \phi & T = \phi \\
  node(T_l', k', T_r') & otherwise
  \end{array}
\right .
\ee

其中

\[
 \begin{array}{l}
 T_l' = map(f, T_l) \\
 T_r' = map(f, T_r) \\
 k' = f(k)
 \end{array}
\]

这一函数可以将一棵树转换成一棵形状完全一样的树,只不过每个节点上的值都按照某个映射进行了变换。我们也可以只访问并操作节点上的值,而不去创建另外一棵树,如下面的C++程序:

\lstset{language=C++}
\begin{lstlisting}
template<class T, class F>
void in_order_walk(node<T>* t, F f) {
  if(t) {
    in_order_walk(t->left, f);
    f(t->value);
    in_order_walk(t->right, f);
  }
}
\end{lstlisting}

这一函数接受一个参数\texttt{f},它可以是一个函数指针,或者是一个函数对象。然后程序按照中序遍历依次应用(apply)f到每个元素上。

我们还可以进一步简化这一算法,通过中序遍历将一棵二叉搜索树转化为一个已序序列。

\be
toList(T) = \left \{
  \begin{array}
  {r@{\quad:\quad}l}
  \phi & T = \phi \\
  toList(T_l) \cup \{ k \} \cup toList(T_r) & otherwise
  \end{array}
\right .
\ee

下面的Haskell程序实现了这一转换函数。

\lstset{language=Haskell}
\begin{lstlisting}
toList::(Ord a)=>Tree a -> [a]
toList Empty = []
toList (Node l x r) = toList l ++ [x] ++ toList r
\end{lstlisting}

我们因此得到了一个排序的方法:先把一个无序的列表转化为一个二叉搜索树,然后再用中序遍历把树转换回列表。这一排序方法被称为“树排序”。记待排序列表为$X = \{x_1, x_2, x_3, ..., x_n\}$。

\be
  sort(X) = toList(fromList(X))
\ee

我们也可以写成函数组合(function composition)的形式:

\[
  sort = toList . fromList
\]

其中函数$fromList$不断地将元素从列表中插入到一棵空的二叉搜索树中。

\be
  fromList(X)= foldL(insert, \phi, X)
\ee

这一表达也可以写成partial application的形式\footnote{亦称为“Curried”形式来纪念数学家和逻辑学家Haskell Curry。}。

\[
  fromList = foldL \quad insert \quad \phi
\]

对于不熟悉从左侧进行folding的读者,这一函数也可以递归地定义如下:

\[
fromList(X) = \left \{
  \begin{array}
  {r@{\quad:\quad}l}
  \phi & X = \phi \\
  insert(fromList(\{x_2, x_3, ..., x_n\}), x_1) & otherwise
  \end{array}
\right .
\]

我们将会大量使用folding、函数组合和parital求值的概念。读者可以参考本书的附录A或其他参考资料,如\cite{wiki-fold}、\cite{func-composition}以及\cite{curry}。

\begin{Exercise}

\begin{itemize}
\item 给定中序遍历和前序遍历的结果,请重建出二叉树,并给出后序遍历的结果。例如:
\begin{itemize}
\item 前序遍历结果:1, 2, 4, 3, 5, 6;
\item 中序遍历结果:4, 2, 1, 5, 3, 6;
\item 后序遍历结果:?
\end{itemize}
\index{tree reconstruction}

\item 编程实现从前序遍历和中序遍历的结果重建二叉树。

\item 证明对二叉搜索树进行中序遍历可以将全部元素按照从小到大的顺序输出。

\item 使用Big-O分析树排序的算法复杂度。
\end{itemize}
\end{Exercise}

% ================================================================
% Querying a binary search tree
% ================================================================
\section{搜索}
\index{二叉搜索树!search}
\index{二叉搜索树!looking up}

二叉搜索树有三种不同的搜索:在树中搜索一个key(亦称look up),查找最大或最小的元素,以及查找给定元素的上一个(predecessor)或下一个(successor)元素。

\subsection{Look up}
二叉搜索树的定义使得它非常适合进行元素的搜索。可以按照下面描述的方法在树中搜索一个key:

\begin{itemize}
\item 如果树为空,搜索失败;
\item 如果根节点的key等于待搜索的值,搜索成功,返回根节点作为结果;
\item 如果待搜索的值小于根节点的key,继续在左子树中递归搜索;
\item 否则,待搜索的值大于根节点的key,继续在右子树中递归搜索。
\end{itemize}

这一算法可以定义为下面的递归函数,其中$T_l$,$T_r$和$k$分别为非空二叉树的左右子树和key。

\be
lookup(T, x) = \left \{
  \begin{array}
  {r@{\quad:\quad}l}
  \phi & T = \phi \\
  T & k = x \\
  lookup(T_l, x) & x < k \\
  lookup(T_r, x) & otherwise
  \end{array}
\right .
\ee

在实际应用中,我们也可以返回这一key对应的数据(satellite data)而不是整个节点。这一算法简单直观,可以直接翻译为下面的Haskell例子程序。

\lstset{language=Haskell}
\begin{lstlisting}
lookup::(Ord a)=> Tree a -> a -> Tree a
lookup Empty _ = Empty
lookup t@(Node l k r) x | k == x = t
                        | x < k = lookup l x
                        | otherwise = lookup r x
\end{lstlisting}

如果进行搜索的二叉树很平衡,绝大多数节点都有非空的左右分支,对于$n$个元素的二叉树,搜索算法的性能为$O(\lg n)$。
If the BST is well balanced, which means that almost
all nodes have both non-NIL left child and right child, for $N$ elements,
the search algorithm takes $O(\lg N)$ time to perform. This is not
formal definition of balance. We'll show it in later post about red-black-tree.
If the tree is poor balanced, the worst case takes $O(N)$ time to
search for a key. If we denote the height of the tree as $h$, we can
uniform the performance of the algorithm as $O(h)$.

The search algorithm can also be realized without using recursion in
a procedural manner.

\begin{algorithmic}[1]
\Function{Search}{$T, x$}
  \While{$T \neq NIL \wedge$ \Call{Key}{$T$} $ \neq x$}
    \If{$x <$ \Call{Key}{$T$}}
      \State $T \gets $ \Call{Left}{$T$}
    \Else
      \State $T \gets $ \Call{Right}{$T$}
    \EndIf
  \EndWhile
  \State \Return $T$
\EndFunction
\end{algorithmic}

Below is the C++ program based on this algorithm.

\lstset{language=C++}
\begin{lstlisting}
template<class T>
node<T>* search(node<T>* t, T x){
  while(t && t->key!=x){
    if(x < t->key) t=t->left;
    else t=t->right;
  }
  return t;
}
\end{lstlisting}

\subsection{Minimum and maximum}
\index{binary search tree!min/max}

Minimum and maximum can be implemented from the property of binary search
tree, less keys are always in left child, and greater keys are in right.

For minimum, we can continue traverse the left sub tree until it is empty.
While for maximum, we traverse the right.

\be
min(T) = \left \{
  \begin{array}
  {r@{\quad:\quad}l}
  key(T) & left(T) = \phi \\
  min(left(T)) & otherwise
  \end{array}
\right .
\ee

\be
max(T) = \left \{
  \begin{array}
  {r@{\quad:\quad}l}
  key(T) & right(T) = \phi \\
  max(right(T)) & otherwise
  \end{array}
\right .
\ee

Both function bound to $O(h)$ time, where $h$ is the height of the tree.
For the balanced BST, $min$/$max$ are bound to $O(\lg N)$ time,
while they are $O(N)$ in the worst cases.

We skip translating them to programs, It's also possible to implement them
in pure procedural way without using recursion.

\subsection{Successor and predecessor}
\index{binary search tree!succ/pred}

The last kind of querying, to find the successor or predecessor of an element
is useful when a tree is treated as a generic container and traversed by
using iterator. It will be relative easier to implement if parent of
a node can be accessed directly.

It seems that the functional solution is hard to be found, because there
is no pointer like field linking to the parent node. One solution is
to left `breadcrumbs' when we visit the tree, and use these information
to back-track or even re-construct the whole tree. Such data structure,
that contains both the tree and `breadcrumbs' is called zipper.
please refer to \cite{zipper-hbook} for details.

However, If we consider
the original purpose of providing $succ$/$pred$ function, `to traverse all the
BST elements one by one` as a generic container, we realize
that they don't make significant sense in functional settings because
we can traverse the tree in increase order by $mapT$ function we defined
previously.

We'll meet many problems in this series of post that they are only valid
in imperative settings, and they are not meaningful problems in functional
settings at all. One good example is how to delete an element in
red-black-tree\cite{okasaki-blog}.

In this section, we'll only present the imperative algorithm for finding
the successor and predecessor in a BST.

When finding the successor of element $x$, which is the smallest one $y$
that satisfies $y > x$, there are two cases. If the node with value $x$
has non-NIL right child, the minimum element in right child is the answer;
For example, in Figure \ref{fig:bst-example}, in order to find the successor
of 8, we search it's right sub tree for the minimum one, which yields 9
as the result. While if node $x$ don't have right child, we need
back-track to find the closest ancestors whose left child is also ancestor
of $x$. In Figure \ref{fig:bst-example}, since 2 don't have right sub tree,
we go back to its parent 1. However, node 1 don't have left child, so we
go back again and reach to node 3, the left child of 3, is also ancestor
of 2, thus, 3 is the successor of node 2.

Based on this description, the algorithm can be given as the following.

\begin{algorithmic}[1]
\Function{Succ}{$x$}
  \If{\Call{Right}{$x$} $\neq NIL$}
    \State \Return \textproc{Min}(\Call{Right}{$x$})
  \Else
    \State $p \gets $ \Call{Parent}{$x$}
    \While{$p \neq NIL$ and $x =$ \Call{Right}{$p$}}
      \State $x \gets p$
      \State $p \gets $ \Call{Parent}{$p$}
    \EndWhile
    \State \Return $p$
  \EndIf
\EndFunction
\end{algorithmic}

The predecessor case is quite similar to the successor algorithm, they
are symmetrical to each other.

\begin{algorithmic}[1]
\Function{Pred}{$x$}
  \If{\Call{Left}{$x$} $\neq NIL$}
    \State \Return \textproc{Max}(\Call{Left}{$x$})
  \Else
    \State $p \gets $ \Call{Parent}{$x$}
    \While{$p \neq NIL$ and $x =$ \Call{Left}{$p$}}
      \State $x \gets p$
      \State $p \gets $ \Call{Parent}{$p$}
    \EndWhile
    \State \Return $p$
  \EndIf
\EndFunction
\end{algorithmic}

Below are the Python programs based on these algorithms. They are changed
a bit in while loop conditions.

\lstset{language=Python}
\begin{lstlisting}
def succ(x):
    if x.right is not None: return tree_min(x.right)
    p = x.parent
    while p is not None and p.left != x:
        x = p
        p = p.parent
    return p

def pred(x):
    if x.left is not None: return tree_max(x.left)
    p = x.parent
    while p is not None and p.right != x:
        x = p
        p = p.parent
    return p
\end{lstlisting}

\begin{Exercise}

\begin{itemize}
\item Can you figure out how to iterate a tree as a generic container
by using $pred()$/$succ()$? What's the performance of such traversing
process in terms of big-O?

\item A reader discussed about traversing all elements inside a
range $[a, b]$. In C++, the algorithm looks like the below code:

$\mathbf{for\_each} (m.lower\_bound(12), m.upper\_bound(26), f);$

Can you provide the purely function solution for this problem?
\index{range traverse}
\end{itemize}

\end{Exercise}

% ================================================================
%                 Deletion
% ================================================================
\section{Deletion}
\index{binary search tree!delete}
Deletion is another `imperative only' topic for binary search tree.
This is because deletion mutate the tree, while in purely functional
settings, we don't modify the tree after building it in most
application.

However, One method of deleting element from binary search
tree in purely functional way is shown in this section. It's actually
reconstructing the tree but not modifying the tree.

Deletion is the most complex operation for binary search tree.
this is because we must keep the BST property, that for any node,
all keys in left sub tree are less than the key of this node, and
they are all less than any keys in right sub tree. Deleting a node
can break this property.

In this post, different with the algorithm described in \cite{CLRS},
A simpler one from SGI STL implementation is used.\cite{sgi-stl}

To delete a node $x$ from a tree.
\begin{itemize}
\item If $x$ has no child or only one child, splice x out;
\item Otherwise ($x$ has two children), use minimum element of its right sub tree to replace $x$, and splice the original minimum element out.
\end{itemize}

The simplicity comes from the truth that, the minimum element is stored
in a node in the right sub tree, which can't have two non-NIL children.
It ends up in the trivial case, the the node can be directly splice
out from the tree.

Figure \ref{fig:del-leaf}, \ref{fig:del-1child}, and \ref{fig:del-branch}
illustrate these different cases when deleting a node from the tree.

\begin{figure}[htbp]
       \begin{center}
	\includegraphics[scale=0.5]{img/del-leaf.ps}
        \caption{$x$ can be spliced out.} \label{fig:del-leaf}
       \end{center}
\end{figure}

\begin{figure}[htbp]
        \centering
        \subfloat[Before delete $x$]{\includegraphics[scale=0.5]{img/del-lc-before.ps}}
        \subfloat[After delete $x$]{\includegraphics[scale=0.5]{img/del-lc-after.ps}} \\
        $x$ is spliced out, and replaced by its left child. \\
        \subfloat[Before delete $x$]{\includegraphics[scale=0.5]{img/del-rc-before.ps}}
        \subfloat[Before delete $x$]{\includegraphics[scale=0.5]{img/del-rc-after.ps}} \\
        $x$ is spliced out, and replaced by its right child.
        \caption{Delete a node which has only one non-NIL child.}
        \label{fig:del-1child}
\end{figure}

\begin{figure}[htbp]
        \centering
        \subfloat[Before delete $x$]{\includegraphics[scale=0.5]{img/del-branch-before.ps}}
        \subfloat[After delete $x$]{ \includegraphics[scale=0.5]{img/del-branch-after.ps}} \\
        $x$ is replaced by splicing the minimum element from its right child.
        \caption{Delete a node which has both children.}
        \label{fig:del-branch}
\end{figure}

Based on this idea, the deletion can be defined as the below function.

\be
delete(T, x) = \left \{
  \begin{array}
  {r@{\quad:\quad}l}
  \phi & T = \phi \\
  node(delete(L, x), K, R) & x < K \\
  node(L, K, delete(R, x)) & x > K \\
  R & x = K \wedge L = \phi \\
  L & x = K \wedge R = \phi \\
  node(L, y, delete(R, y)) & otherwise
  \end{array}
\right .
\ee

Where
\[
\begin{array}{l}
L = left(T) \\
R = right(T) \\
K = key(T) \\
y = min(R)
\end{array}
\]

Translating the function to Haskell yields the below program.

\lstset{language=Haskell}
\begin{lstlisting}
delete::(Ord a)=> Tree a -> a -> Tree a
delete Empty _ = Empty
delete (Node l k r) x | x < k = (Node (delete l x) k r)
                      | x > k = (Node l k (delete r x))
                      -- x == k
                      | isEmpty l = r
                      | isEmpty r = l
                      | otherwise = (Node l k' (delete r k'))
                          where k' = min r
\end{lstlisting}

Function `isEmpty' is used to test if a tree is empty ($\phi$).
Note that the algorithm first performs search to locate the node
where the element need be deleted, after that it execute the
deletion. This algorithm takes $O(h)$ time where $h$ is the height
of the tree.

It's also possible to pass the node but not the element to the
algorithm for deletion. Thus the searching is no more needed.

The imperative algorithm is more complex because it need set the
parent properly. The function will return the root of the result tree.

\begin{algorithmic}[1]
\Function{Delete}{$T, x$}
  \State $root \gets T$
  \State $x' \gets x$ \Comment{save $x$}
  \State $parent \gets $ \Call{Parent}{$x$}
  \If{\Call{Left}{$x$} $= NIL$}
    \State $x \gets $ \Call{Right}{$x$}
  \ElsIf{\Call{Right}{$x$} $= NIL$}
    \State $x \gets $ \Call{Left}{$x$}
  \Else
    \Comment{both children are non-NIL}
    \State  $y \gets $ \textproc{Min}(\Call{Right}{$x$})
    \State \Call{Key}{$x$} $\gets$ \Call{Key}{$y$}
    \State Copy other satellite data from $y$ to $x$
    \If{\Call{Parent}{$y$} $\neq x$}
      \Comment{$y$ hasn't left sub tree}
      \State \textproc{Left}(\Call{Parent}{$y$}) $\gets$ \Call{Right}{$y$}
    \Else
      \Comment{$y$ is the root of right child of $x$}
      \State \Call{Right}{$x$} $\gets$ \Call{Right}{$y$}
    \EndIf
    \State Remove $y$
    \State \Return $root$
  \EndIf
  \If{$x \neq NIL$}
    \State \Call{Parent}{$x$} $\gets parent$
  \EndIf
  \If{$parent = NIL$}
    \Comment{We are removing the root of the tree}
    \State $root \gets x$
  \Else
    \If{\Call{Left}{$parent$} $= x'$}
      \State \Call{Left}{$parent$} $\gets x$
    \Else
      \State \Call{Right}{$parent$} $\gets x$
    \EndIf
  \EndIf
  \State Remove $x'$
  \State \Return $root$
\EndFunction
\end{algorithmic}

Here we assume the node to be deleted is not empty (otherwise we can
simply returns the original tree). In other cases, it will first record
the root of the tree, create copy pointers to $x$, and its parent.

If either of the children is empty, the algorithm just splice $x$ out.
If it has two non-NIL children, we first located the minimum of right
child, replace the key of $x$ to $y$'s, copy the satellite data as
well, then splice $y$ out. Note that there is a special case that $y$
is the root node of $x$'s left sub tree.

Finally we need reset the stored parent if the original $x$ has only
one non-NIL child.
If the parent pointer we copied before is empty, it
means that we are deleting the root node, so we need return the new root. After
the parent is set properly, we finally remove the old x from memory.

The relative Python program for deleting algorithm is given as below.
Because Python provides GC, we needn't explicitly remove the node
from the memory.

\lstset{language=Python}
\begin{lstlisting}
def tree_delete(t, x):
    if x is None:
        return t
    [root, old_x, parent] = [t, x, x.parent]
    if x.left is None:
        x = x.right
    elif x.right is None:
        x = x.left
    else:
        y = tree_min(x.right)
        x.key = y.key
        if y.parent != x:
            y.parent.left = y.right
        else:
            x.right = y.right
        return root
    if x is not None:
        x.parent = parent
    if parent is None:
        root = x
    else:
        if parent.left == old_x:
            parent.left = x
        else:
            parent.right = x
    return root
\end{lstlisting}

Because the procedure seeks minimum element, it runs in $O(h)$ time on
a tree of height $h$.

\begin{Exercise}

\begin{itemize}
\item There is a symmetrical solution for deleting a node which has two
non-NIL children, to replace the element by splicing the maximum one out
off the left sub-tree. Write a program to implement this solution.
\end{itemize}

\end{Exercise}

\section{Randomly build binary search tree}
\index{binary search tree!randomly build}
It can be found that all operations given in this post bound to $O(h)$
time for a tree of height $h$. The height affects the performance
a lot. For a very unbalanced tree, $h$ tends to be $O(N)$, which leads
to the worst case. While for balanced tree, $h$ close to $O(\lg N)$.
We can gain the good performance.

How to make the binary search tree
balanced will be discussed in next post. However, there exists a simple
way. Binary search tree can be randomly built as described in \cite{CLRS}.
Randomly building can help to avoid (decrease the possibility) unbalanced
binary trees. The idea is that before building the tree, we can call a random
process, to shuffle the elements.

\begin{Exercise}

\begin{itemize}
\item Write a randomly building process for binary search tree.
\end{itemize}

\end{Exercise}

\section{Appendix} \label{appendix}
%\appendix
All programs are provided along with this post. They are free for downloading.
We provided C, C++, Python, Haskell, and Scheme/Lisp programs as example.

\begin{thebibliography}{99}

\bibitem{CLRS}
Thomas H. Cormen, Charles E. Leiserson, Ronald L. Rivest and Clifford Stein.
``Introduction to Algorithms, Second Edition''. ISBN:0262032937. The MIT Press. 2001

\bibitem{Bentley}
Jon Bentley. ``Programming Pearls(2nd Edition)''. Addison-Wesley Professional; 2 edition (October 7, 1999). ISBN-13: 978-0201657883

\bibitem{okasaki-blog}
Chris Okasaki. ``Ten Years of Purely Functional Data Structures''. http://okasaki.blogspot.com/2008/02/ten-years-of-purely-functional-data.html

\bibitem{sgi-stl}
SGI. ``Standard Template Library Programmer's Guide''. http://www.sgi.com/tech/stl/

\bibitem{literal-program}
http://en.literateprograms.org/Category:Binary\_search\_tree

\bibitem{wiki-fold}
http://en.wikipedia.org/wiki/Foldl

\bibitem{func-composition}
http://en.wikipedia.org/wiki/Function\_composition

\bibitem{curry}
http://en.wikipedia.org/wiki/Partial\_application

\bibitem{zipper-hbook}
Miran Lipovaca. ``Learn You a Haskell for Great Good! A Beginner's Guide''. the last chapter. No Starch Press; 1 edition April 2011, 400 pp. ISBN: 978-1-59327-283-8

\end{thebibliography}

\ifx\wholebook\relax\else
\end{document}
\fi
