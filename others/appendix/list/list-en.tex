\ifx\wholebook\relax \else
% ------------------------ 

\documentclass{article}
%------------------- Other types of document example ------------------------
%
%\documentclass[twocolumn]{IEEEtran-new}
%\documentclass[12pt,twoside,draft]{IEEEtran}
%\documentstyle[9pt,twocolumn,technote,twoside]{IEEEtran}
%
%-----------------------------------------------------------------------------
%%
% loading packages
%
\newif\ifpdf
\ifx\pdfoutput\undefined % We're not running pdftex
  \pdffalse
\else
  \pdftrue
\fi
%
%
\ifpdf
  \RequirePackage[pdftex,%
            CJKbookmarks,%
       bookmarksnumbered,%
              colorlinks,%
          linkcolor=blue,%
              hyperindex,%
        plainpages=false,%
       pdfstartview=FitH]{hyperref}
\else
  \RequirePackage[dvipdfm,%
             CJKbookmarks,%
        bookmarksnumbered,%
               colorlinks,%
           linkcolor=blue,%
               hyperindex,%
         plainpages=false,%
        pdfstartview=FitH]{hyperref}
  \AtBeginDvi{\special{pdf:tounicode GBK-EUC-UCS2}} % GBK -> Unicode
\fi
\usepackage{hyperref}

% other packages
%-----------------------------------------------------------------------------
\usepackage{graphicx, color}
\usepackage{CJK}
%
% for programming 
%
\usepackage{verbatim}
\usepackage{listings}


\lstdefinelanguage{Smalltalk}{
  morekeywords={self,super,true,false,nil,thisContext}, % This is overkill
  morestring=[d]',
  morecomment=[s]{"}{"},
  alsoletter={\#:},
  escapechar={!},
  literate=
    {BANG}{!}1
    {UNDERSCORE}{\_}1
    {\\st}{Smalltalk}9 % convenience -- in case \st occurs in code
    % {'}{{\textquotesingle}}1 % replaced by upquote=true in \lstset
    {_}{{$\leftarrow$}}1
    {>>>}{{\sep}}1
    {^}{{$\uparrow$}}1
    {~}{{$\sim$}}1
    {-}{{\sf -\hspace{-0.13em}-}}1  % the goal is to make - the same width as +
    %{+}{\raisebox{0.08ex}{+}}1		% and to raise + off the baseline to match -
    {-->}{{\quad$\longrightarrow$\quad}}3
	, % Don't forget the comma at the end!
  tabsize=2
}[keywords,comments,strings]

\lstloadlanguages{C++, Lisp, Smalltalk}

% ======================================================================

\def\BibTeX{{\rm B\kern-.05em{\sc i\kern-.025em b}\kern-.08em
    T\kern-.1667em\lower.7ex\hbox{E}\kern-.125emX}}

\newtheorem{theorem}{Theorem}

%
% mathematics
%
\newcommand{\be}{\begin{equation}}
\newcommand{\ee}{\end{equation}}
\newcommand{\bmat}[1]{\left( \begin{array}{#1} }
\newcommand{\emat}{\end{array} \right) }
\newcommand{\VEC}[1]{\mbox{\boldmath $#1$}}

% numbered equation array
\newcommand{\bea}{\begin{eqnarray}}
\newcommand{\eea}{\end{eqnarray}}

% equation array not numbered
\newcommand{\bean}{\begin{eqnarray*}}
\newcommand{\eean}{\end{eqnarray*}}

\RequirePackage{CJK,CJKnumb,CJKulem,CJKpunct}
% we use CJK as default environment
\AtBeginDocument{\begin{CJK*}{GBK}{song}\CJKtilde\CJKindent\CJKcaption{GB}}
\AtEndDocument{\clearpage\end{CJK*}}

%
% loading packages
%
\newif\ifpdf
\ifx\pdfoutput\undefined % We're not running pdftex
  \pdffalse
\else
  \pdftrue
\fi
%
%
\ifpdf
  \RequirePackage[pdftex,%
       bookmarksnumbered,%
              colorlinks,%
          linkcolor=blue,%
              hyperindex,%
        plainpages=false,%
       pdfstartview=FitH]{hyperref}
\else
  \RequirePackage[dvipdfm,%
        bookmarksnumbered,%
               colorlinks,%
           linkcolor=blue,%
               hyperindex,%
         plainpages=false,%
        pdfstartview=FitH]{hyperref}
\fi
\usepackage{hyperref}

% other packages
%-----------------------------------------------------------------------------
\usepackage{graphicx, color}
%
% for programming 
%
\usepackage{verbatim}
\usepackage{listings}
\usepackage{algorithmic} %for pseudocode
\usepackage{algorithm}


\lstdefinelanguage{Smalltalk}{
  morekeywords={self,super,true,false,nil,thisContext}, % This is overkill
  morestring=[d]',
  morecomment=[s]{"}{"},
  alsoletter={\#:},
  escapechar={!},
  literate=
    {BANG}{!}1
    {UNDERSCORE}{\_}1
    {\\st}{Smalltalk}9 % convenience -- in case \st occurs in code
    % {'}{{\textquotesingle}}1 % replaced by upquote=true in \lstset
    {_}{{$\leftarrow$}}1
    {>>>}{{\sep}}1
    {^}{{$\uparrow$}}1
    {~}{{$\sim$}}1
    {-}{{\sf -\hspace{-0.13em}-}}1  % the goal is to make - the same width as +
    %{+}{\raisebox{0.08ex}{+}}1		% and to raise + off the baseline to match -
    {-->}{{\quad$\longrightarrow$\quad}}3
	, % Don't forget the comma at the end!
  tabsize=2
}[keywords,comments,strings]

\lstloadlanguages{C++, Lisp, Haskell, Python, Smalltalk}

% ======================================================================

\def\BibTeX{{\rm B\kern-.05em{\sc i\kern-.025em b}\kern-.08em
    T\kern-.1667em\lower.7ex\hbox{E}\kern-.125emX}}

\newtheorem{theorem}{Theorem}

%
% mathematics
%
\newcommand{\be}{\begin{equation}}
\newcommand{\ee}{\end{equation}}
\newcommand{\bmat}[1]{\left( \begin{array}{#1} }
\newcommand{\emat}{\end{array} \right) }
\newcommand{\VEC}[1]{\mbox{\boldmath $#1$}}

% numbered equation array
\newcommand{\bea}{\begin{eqnarray}}
\newcommand{\eea}{\end{eqnarray}}

% equation array not numbered
\newcommand{\bean}{\begin{eqnarray*}}
\newcommand{\eean}{\end{eqnarray*}}




\setcounter{page}{1}

\begin{document}

\fi
%--------------------------

% ================================================================
%                 COVER PAGE
% ================================================================

\title{Lists}

\author{Liu~Xinyu
\thanks{{\bfseries Liu Xinyu } \newline
  Email: liuxinyu95@gmail.com \newline}
  }

\markboth{Sequences}{AlgoXY}

\maketitle

\ifx\wholebook\relax
\chapter{Lists}
\numberwithin{Exercise}{chapter}
\fi

% ================================================================
%                 Introduction
% ================================================================
\section{Introduction}
\label{introduction}
This book intensely uses recursive list manipulations in purely functional settings.
List can be treated as a counterpart to arrays in imperative settings, which is
bricks to many algorithms and data structures.

For the readers who are not familiar with functional list manipulation, this appendix
provides a quick reference. All operations listed in this appendix are not only
described in equations, but also implemented in both functional programming languages
as well as imperative languages examples. We also provide a special type of
implementation in C++ template meta programming similar to \cite{moderncxx}
for interesting.

Besides the elementary list operations, this appendix also contains explaination of 
some high order function concepts such as mapping, folding etc.


% ================================================================
%                 Binary random access list
% ================================================================
\section{List Definition}
\index{List!Definition}

Like arrays in imperative settings, lists play a critical role in functional setting\footnote{Some 
reader may argue that `lambda calculus plays the most critical role'.
Lambda calculus is somewhat as assembly languages to the computation world, which 
is worthy studying from the essense of computation model to the practical programs.
However, we don't dive into the topic in this book. Users can refer to \cite{mittype}
for detail.}. Lists are built-in supported in some programming languages like Lisp
families and ML families so it needn't explicitly define list in those environment.

List, or more precisely, singly linked-list is a data strucutre that can be described
below.

\begin{itemize}
\item A {\em list} is either empty;
\item Or contains an element and a {\em list}.
\end{itemize}

Note that this definition is recursive. Figure \ref{fig:list-example} illustrates
a list with $N$ nodes. Each node contains two part, a key element and a sub list. The
sub list contained in the last node is empty, which is denoted as 'NIL'.

\begin{figure}[htbp]
        \centering
        \includegraphics[scale=0.8]{img/list-example.ps}
        \caption{A list contains $N$ nodes} \label{fig:list-example}
\end{figure}

This data structure can be explictly defined in programming languages support record
(or compound type) concept. The following ISO C++ code defines list.

\lstset{language=C++}
\begin{lstlisting}
template<typename T>
struct List {
  T key;
  List* next;
};
\end{lstlisting}

Given a list $L$, two functions can be defined to access the element stored in it and
the sub list respectively. They are typically denoted as $first(L)$, and $rest(L)$.
These two functions are typically named as \verb|car| and \verb|cdr| in Lisp for historic reason
about the design of machine registers \cite{SICP}. In languages support Pattern matching (e.g. ML families)
These two functions are commonly realized by matching the \verb|cons| which we'll introduced
later. for example the following Haskell program:

\lstset{language=Haskell}
\begin{lstlisting}
head (x:xs) = x
tail (x:xs) = xs
\end{lstlisting}

If the list is defined in record syntax like what we did above, these two functions can 
be realized by accessing the record fields \footnote{They can be also named as 'key' and 'next'
or be defined as class methods.}.

\lstset{language=C++}
\begin{lstlisting}
template<typename T>
T first(List<T> *xs) { return xs->key; }

template<typename T>
List<T>* rest(List<T>* xs) { return xs->next; }
\end{lstlisting}

More interesting, as far as in an environment support recursion, we can define List. The following
example define a list of integers in C++ compile time.

\lstset{language=C++}
\begin{lstlisting}
struct Empty;

template<n, T> struct List {
  static const int first = n;
  typedef T rest;
};
\end{lstlisting}

These line construct a list of $\{1, 2, 3, 4, 5\}$ in compile time.

\begin{lstlisting}
typedef List<1, List<2, List<3, List<4 List<5, Empty> > > > > A;
\end{lstlisting}

\section{Basic list manipulation}

\subsection{Construction}

% ================================================================
%                 Short summary
% ================================================================
\section{Notes and short summary}
...

% ================================================================
%                 Appendix
% ================================================================

\begin{thebibliography}{99}

\bibitem{moderncxx}
Andrei Alexandrescu. ``Modern C++ design: Generic Programming and Design Patterns Applied''. Addison Wesley February 01, 2001, ISBN 0-201-70431-5

\bibitem{mittype}
Benjamin C. Pierce. ``Types and Programming Languages''. The MIT Press, 2002. ISBN:0262162091

\bibitem{SICP}
Harold Abelson, Gerald Jay Sussman, Julie Sussman. ``Structure and Interpretation of Computer Programs, 2nd Edition''. MIT Press, 1996, ISBN 0-262-51087-1

\end{thebibliography}

\ifx\wholebook\relax \else
\end{document}
\fi

% LocalWords:  typedef struct typename
