\ifx\wholebook\relax \else
% ------------------------ 

\documentclass{article}
%------------------- Other types of document example ------------------------
%
%\documentclass[twocolumn]{IEEEtran-new}
%\documentclass[12pt,twoside,draft]{IEEEtran}
%\documentstyle[9pt,twocolumn,technote,twoside]{IEEEtran}
%
%-----------------------------------------------------------------------------
%%
% loading packages
%
\newif\ifpdf
\ifx\pdfoutput\undefined % We're not running pdftex
  \pdffalse
\else
  \pdftrue
\fi
%
%
\ifpdf
  \RequirePackage[pdftex,%
            CJKbookmarks,%
       bookmarksnumbered,%
              colorlinks,%
          linkcolor=blue,%
              hyperindex,%
        plainpages=false,%
       pdfstartview=FitH]{hyperref}
\else
  \RequirePackage[dvipdfm,%
             CJKbookmarks,%
        bookmarksnumbered,%
               colorlinks,%
           linkcolor=blue,%
               hyperindex,%
         plainpages=false,%
        pdfstartview=FitH]{hyperref}
  \AtBeginDvi{\special{pdf:tounicode GBK-EUC-UCS2}} % GBK -> Unicode
\fi
\usepackage{hyperref}

% other packages
%-----------------------------------------------------------------------------
\usepackage{graphicx, color}
\usepackage{CJK}
%
% for programming 
%
\usepackage{verbatim}
\usepackage{listings}


\lstdefinelanguage{Smalltalk}{
  morekeywords={self,super,true,false,nil,thisContext}, % This is overkill
  morestring=[d]',
  morecomment=[s]{"}{"},
  alsoletter={\#:},
  escapechar={!},
  literate=
    {BANG}{!}1
    {UNDERSCORE}{\_}1
    {\\st}{Smalltalk}9 % convenience -- in case \st occurs in code
    % {'}{{\textquotesingle}}1 % replaced by upquote=true in \lstset
    {_}{{$\leftarrow$}}1
    {>>>}{{\sep}}1
    {^}{{$\uparrow$}}1
    {~}{{$\sim$}}1
    {-}{{\sf -\hspace{-0.13em}-}}1  % the goal is to make - the same width as +
    %{+}{\raisebox{0.08ex}{+}}1		% and to raise + off the baseline to match -
    {-->}{{\quad$\longrightarrow$\quad}}3
	, % Don't forget the comma at the end!
  tabsize=2
}[keywords,comments,strings]

\lstloadlanguages{C++, Lisp, Smalltalk}

% ======================================================================

\def\BibTeX{{\rm B\kern-.05em{\sc i\kern-.025em b}\kern-.08em
    T\kern-.1667em\lower.7ex\hbox{E}\kern-.125emX}}

\newtheorem{theorem}{Theorem}

%
% mathematics
%
\newcommand{\be}{\begin{equation}}
\newcommand{\ee}{\end{equation}}
\newcommand{\bmat}[1]{\left( \begin{array}{#1} }
\newcommand{\emat}{\end{array} \right) }
\newcommand{\VEC}[1]{\mbox{\boldmath $#1$}}

% numbered equation array
\newcommand{\bea}{\begin{eqnarray}}
\newcommand{\eea}{\end{eqnarray}}

% equation array not numbered
\newcommand{\bean}{\begin{eqnarray*}}
\newcommand{\eean}{\end{eqnarray*}}

\RequirePackage{CJK,CJKnumb,CJKulem,CJKpunct}
% we use CJK as default environment
\AtBeginDocument{\begin{CJK*}{GBK}{song}\CJKtilde\CJKindent\CJKcaption{GB}}
\AtEndDocument{\clearpage\end{CJK*}}

%
% loading packages
%
\newif\ifpdf
\ifx\pdfoutput\undefined % We're not running pdftex
  \pdffalse
\else
  \pdftrue
\fi
%
%
\ifpdf
  \RequirePackage[pdftex,%
       bookmarksnumbered,%
              colorlinks,%
          linkcolor=blue,%
              hyperindex,%
        plainpages=false,%
       pdfstartview=FitH]{hyperref}
\else
  \RequirePackage[dvipdfm,%
        bookmarksnumbered,%
               colorlinks,%
           linkcolor=blue,%
               hyperindex,%
         plainpages=false,%
        pdfstartview=FitH]{hyperref}
\fi
\usepackage{hyperref}

% other packages
%-----------------------------------------------------------------------------
\usepackage{graphicx, color}
%
% for programming 
%
\usepackage{verbatim}
\usepackage{listings}
\usepackage{algorithmic} %for pseudocode
\usepackage{algorithm}


\lstdefinelanguage{Smalltalk}{
  morekeywords={self,super,true,false,nil,thisContext}, % This is overkill
  morestring=[d]',
  morecomment=[s]{"}{"},
  alsoletter={\#:},
  escapechar={!},
  literate=
    {BANG}{!}1
    {UNDERSCORE}{\_}1
    {\\st}{Smalltalk}9 % convenience -- in case \st occurs in code
    % {'}{{\textquotesingle}}1 % replaced by upquote=true in \lstset
    {_}{{$\leftarrow$}}1
    {>>>}{{\sep}}1
    {^}{{$\uparrow$}}1
    {~}{{$\sim$}}1
    {-}{{\sf -\hspace{-0.13em}-}}1  % the goal is to make - the same width as +
    %{+}{\raisebox{0.08ex}{+}}1		% and to raise + off the baseline to match -
    {-->}{{\quad$\longrightarrow$\quad}}3
	, % Don't forget the comma at the end!
  tabsize=2
}[keywords,comments,strings]

\lstloadlanguages{C++, Lisp, Haskell, Python, Smalltalk}

% ======================================================================

\def\BibTeX{{\rm B\kern-.05em{\sc i\kern-.025em b}\kern-.08em
    T\kern-.1667em\lower.7ex\hbox{E}\kern-.125emX}}

\newtheorem{theorem}{Theorem}

%
% mathematics
%
\newcommand{\be}{\begin{equation}}
\newcommand{\ee}{\end{equation}}
\newcommand{\bmat}[1]{\left( \begin{array}{#1} }
\newcommand{\emat}{\end{array} \right) }
\newcommand{\VEC}[1]{\mbox{\boldmath $#1$}}

% numbered equation array
\newcommand{\bea}{\begin{eqnarray}}
\newcommand{\eea}{\end{eqnarray}}

% equation array not numbered
\newcommand{\bean}{\begin{eqnarray*}}
\newcommand{\eean}{\end{eqnarray*}}




\setcounter{page}{1}

\begin{document}

\fi
%--------------------------

% ================================================================
%                 COVER PAGE
% ================================================================

\title{The evolution of selection sort}

\author{Liu~Xinyu
\thanks{{\bfseries Liu Xinyu } \newline
  Email: liuxinyu95@gmail.com \newline}
  }

\markboth{The evolution of selection sort}{AlgoXY}

\maketitle

\ifx\wholebook\relax
\chapter{The evolution of selection sort}
\numberwithin{Exercise}{chapter}
\fi

% ================================================================
%                 Introduction
% ================================================================
\section{Introduction}
\label{introduction} \index{selection sort}
We have introduced the `hello world' sorting algorithm, insertion
sort. In this short chapter, we explain another straightforward
sorting method, selection sort. The basic version of
selection sort doesn't perform as good as the divide and conqueror
methods, e.g. quick sort and merge sort. We'll use
the same approaches in the chapter of insertion sort, to 
analyze why it's slow, and try to improve it by vaires of
attempts till reach the best bound of comparison
based sorting, $O(N \lg N)$, by evolving to heap sort.

The idea of selection sort can be illustrated by a real life
story. Consider a kid eating a bunch of grapse. There are two
types of children according to my observation. One is optimistic
type, that the kid always eats the biggest grape he/she can ever
find; the other is pessimistic, that he/she always eats the
smallest one.

The first type of kids actually eat the grape in an order that
the size decreases monotically; while the other eat in a increase
order. The kids {\em sort} the grapes in order of size in fact,
and the method used here is selection sort.

TODO: add a picture to illustrate this.

Based on this idea, the algorithm of selection sort can be directly
described as the following. 

In order to sort a series of elements:

\begin{itemize}
\item The trivial case, if the series is empty, then we are done, the result is also empty;
\item Otherwise, we find the smallest element, and append it to the tail of the result;
\end{itemize}

Note that this algorithm sorts the elements in increase order; It's easy to sort in decrease
order by picking the biggest element intead; We'll introduce about passing a comparator as
parameter later on.

This description can be formalized to a equation.

\be
sort(A) =  \left \{
  \begin{array}
  {r@{\quad:\quad}l}
  \Phi & A = \Phi \\
  \{ m \} \cup sort(A') & otherwise
  \end{array}
\right.  
\ee

Where $m$ is the minimum elenets among collection $A$, and $A'$ is all the rest elements
except $m$:

\[
\begin{array}{l}
m = min(A) \\
A' = A - \{ m \}
\end{array}
\]

We don't limit the data structure of the collection here. Typically, $A$ is array in imperative
environment, and list (singly linked-list particularly) in functional environment, and it can 
be even others as we introduced later.

The algorithm can also be given in imperative manner.

\begin{algorithmic}
\Function{Sort}{$A$}
  \State $X \gets \Phi$
  \While{$A \neq \Phi$}
    \State $x \gets$ \Call{Min}{$A$}
    \State $A \gets$ \Call{Del}{$A, x$}
    \State $X \gets$ \Call{Append}{$X, x$}
  \EndWhile
  \State \Return $X$
\EndFunction
\end{algorithmic}

TODO: Add a figure to illustrate the algorithm.

We just translate the very original idea of `eating grapes' line by line without
considering any expense of time and space cost. This realization store the result
in $X$, and when an selected element is appended to $X$, we delete the same element
from $A$. This indicates that we can change it to `in-place' sorting to reuse
the spaces in $A$.

The idea is to store the minimum element in the first cell in $A$ (we use term `cell' if
$A$ is array, and say `node' if $A$ is list); then store the second minimum element
in next cell, then third cell, ... 

One solution to realize this sorting strategy is swapping. When we select the $i$-th 
minimum element, we swap it with the element in the $i$-th cell:

\begin{algorithmic}
\Function{Sort}{$A$}
  \For{$i \gets 1$ to $|A|$}
    \State $m \gets$ \Call{Min}{$A$}
    \State \textproc{Exchange} $A[i] \leftrightarrow m$
  \EndFor
\EndFunction
\end{algorithmic}

TODO: stop here...

At any time, when we process the $i$-th element, all elements before $i$
have already been sorted. we continuously insert the current elements
until consume all the unsorted data. This idea is illustrated as in figure
\ref{fig:in-place-sort}.

%\begin{figure}[htbp]
%  \centering
%  \includegraphics[scale=0.8]{img/in-place-sort.ps}
%  \caption{The left part is sorted data, continuously insert elements to sorted part.} 
%  \label{fig:in-place-sort}
%\end{figure}

We can find there is recursive concept in this definition. Thus it can
be expressed as the following.

\be
sort(A) = \left \{
  \begin{array}
  {r@{\quad:\quad}l}
  \Phi & A = \Phi \\
  insert(sort(\{A_2, A_3, ...\}), A_1) & otherwise
  \end{array}
\right.  
\ee

% ================================================================
% Insertion
% ================================================================
\section{Insertion}
\index{insertion sort!insertion}

We haven't answered the question about how to realize insertion however.
It's a puzzle how does human locate the proper position so quickly.

For computer, it's an obvious option to perform a scan. We can either 
scan from left to right or vice versa. However, if the sequence is
stored in plain array, it's necessary to scan from right to left.

\begin{algorithmic}
\Function{Sort}{$A$}
  \For{$i \gets 2$ to \Call{Length}{$A$}}
    \Comment{Insert $A[i]$ to sorted sequence $A[1...i-1]$}
    \State $x \gets A[i]$
    \State $j \gets i-1$
    \While{$j > 0 \land x < A[j]$ }
      \State $A[j+1] \gets A[j]$
      \State $j \gets j - 1$
    \EndWhile
    \State $A[j+1] \gets x$
  \EndFor
\EndFunction
\end{algorithmic}

One may think scan from left to right is natural. However, it isn't
as effect as above algorithm for plain array. The reason is that, it's
expensive to insert an element in arbitrary position in an array.
As array stores elements continuously. If we want to insert new element
$x$ in position $i$, we must shift all elements after $i$, including
$i+1, i+2, ...$ one position to right. After that the cell at position $i$
is empty, and we can put $x$ in it. This is illustrated in
figure \ref{fig:array-shift}.

%\begin{figure}[htbp]
%  \centering
%  \includegraphics[scale=0.8]{img/array-shift.ps}
%  \caption{Insert $x$ to array $A$ at position $i$.}
%  \label{fig:array-shift}
%\end{figure}

If the length of array is $N$, this indicates we need examine the 
first $i$ elements, then perform
$N-i+1$ moves, and then insert $x$ to the $i$-th cell. So insertion
from left to right need traverse the whole array anyway. 
While if we scan from right to
left, we totally examine the last $j = N-i+1$ elements, and perform the same
amount of moves. If $j$ is small (e.g. less than $N/2$), there is possibility
to perform less operations than scan from left to right.

Translate the above algorithm to Python yields the following code.

\lstset{language=Python}
\begin{lstlisting}
def isort(xs):
    n = len(xs)
    for i in range(1, n):
        x = xs[i]
        j = i - 1
        while j >= 0 and x < xs[j]:
            xs[j+1] = xs[j]
            j = j - 1
        xs[j+1] = x
\end{lstlisting}

It can be found some other equivalent programs, for instance the following
ANSI C program. However this version isn't as effective as the pseudo code.

\lstset{language=C}
\begin{lstlisting}
void isort(Key* xs, int n){
  int i, j;
  for(i=1; i<n; ++i)
    for(j=i-1; j>=0 && xs[j+1] < xs[j]; --j)
      swap(xs, j, j+1);
}
\end{lstlisting}

This is because the swapping function, which can exchange two elements
typically uses a temporary variable like the following:

\begin{lstlisting}
void swap(Key* xs, int i, int j){
  Key temp = xs[i];
  xs[i] = xs[j];
  xs[j] = temp;
}
\end{lstlisting}

So the ANSI C program presented above takes $3M$ times assignment, where $M$
is the number of inner loops. While the pseudo code as well as the Python
program use shift operation instead of swapping. There are $N+2$ times 
assignment.

We can also provide \textproc{Insert}() function explicitly, and call it
from the general insertion sort algorithm in previous section. We skip 
the detailed realization here and left it as an exercise.

All the insertion algorithms are bound to $O(N)$, where $N$ is the length of
the sequence. No matter what difference among them, such as scan from left
or from right. Thus the over all performance for insertion sort is quadratic
as $O(N^2)$.

\begin{Exercise}

\begin{itemize}
\item Provide explicit insertion function, and call it with general
insertion sort algorithm. Please realize it in both procedural way and
functional way.
\end{itemize}

\end{Exercise}

% ================================================================
% Improvement 1
% ================================================================

\section{Improvement 1}
\index{Insertion sort!binary search}

Let's go back to the question, that why human being can find the proper
position for insertion so quickly. We have shown a solution based on scan.
Note the fact that at any time, all cards at hands have been well sorted,
another possible solution is to use binary search to find that location.

We'll explain the search algorithms in other dedicated chapter. Binary
search is just briefly introduced for illustration purpose here.

The algorithm will be changed to call a binary search procedure.

\begin{algorithmic}
\Function{Sort}{$A$}
  \For{$i \gets 2$ to \Call{Length}{$A$}}
    \State $x \gets A[i]$
    \State $p \gets $ \Call{Binary-Search}{$A[1...i-1], x$}
    \For{$j \gets i$ down to $p$}
      \State $A[j] \gets A[j-1]$
    \EndFor
    \State $A[p] \gets x$
  \EndFor
\EndFunction
\end{algorithmic}

Instead of scan elements one by one, binary search utilize the information
that all elements in slice of array $\{A_1, ..., A_{i-1} \}$ are sorted. 
Let's assume
the order is monotonic increase order. To find a position $j$ that satisfies
$A_{j-1} \leq x \leq A_{j}$. We can first examine the middle element, for 
example, $A_{\lfloor i/2 \rfloor}$. If $x$ is less than it, we need next recursively
perform binary search in the first half of the sequence; otherwise, we
only need search in last half.

Every time, we halve the elements to be examined, this search process runs
$O(\lg N)$ time to locate the insertion position.

\begin{algorithmic}
\Function{Binary-Search}{$A, x$}
  \State $l \gets 1$
  \State $u \gets 1+$ \Call{Length}{$A$}
  \While{$l < u$}
    \State $m \gets \lfloor \frac{l+u}{2} \rfloor$
    \If{$A_m = x$}
      \State \Return $m$ \Comment{Find a duplicated element}
    \ElsIf{$A_m < x$}
      \State $l \gets m+1$
    \Else
      \State $u \gets m$
    \EndIf
  \EndWhile
  \State \Return $l$
\EndFunction
\end{algorithmic}

The improved insertion sort algorithm is still bound to $O(N^2)$, 
compare to previous section, which we use $O(N^2)$ times comparison and
$O(N^2)$ moves, with binary search, we just use $O(N \lg N)$ times 
comparison and $O(N^2)$ moves.

The Python program regarding to this algorithm is given below.

\lstset{language=Python}
\begin{lstlisting}
def isort(xs):
    n = len(xs)
    for i in range(1, n):
        x = xs[i]
        p = binary_search(xs[:i], x)
        for j in range(i, p, -1):
            xs[j] = xs[j-1]
        xs[p] = x

def binary_search(xs, x):
    l = 0
    u = len(xs)
    while l < u:
        m = (l+u)/2
        if xs[m] == x:
            return m 
        elif xs[m] < x:
            l = m + 1
        else:
            u = m
    return l
\end{lstlisting}

\begin{Exercise}
Write the binary search in recursive manner. You needn't use purely functional
programming language.
\end{Exercise}

% ================================================================
% Improvement 2
% ================================================================

\section{Improvement 2}
\index{Insertion sort!linked-list setting}

Although we improve the search time to $O(N \lg N)$ in previous section, the
number of moves is still $O(N^2)$. The reason of why movement takes so long
time, is because the sequence is stored in plain array. The nature of array
is continuously layout data structure, so the insertion operation is expensive.
This hints us that we can use linked-list setting to represent the sequence.
It can improve the insertion operation from $O(N)$ to constant time $O(1)$.

\be
  insert(A, x) = \left \{
  \begin{array}
  {r@{\quad:\quad}l}
  \{ x \} & A = \Phi \\
  \{ x \} \cup A & x < A_1 \\
  \{ A_1 \} \cup insert(\{ A_2, A_3, ... A_n\}, x)& otherwise
  \end{array}
\right.  
\ee

Translating the algorithm to Haskell yields the below program.

\lstset{language=Haskell}
\begin{lstlisting}
insert :: (Ord a) => [a] -> a -> [a]
insert [] x = [x]
insert (y:ys) x = if x < y then x:y:ys else y:insert ys x
\end{lstlisting}

And we can complete the two versions of insertion sort program based on
the first two equations in this chapter.

\begin{lstlisting}
isort [] = []
isort (x:xs) = insert (isort xs) x
\end{lstlisting}

Or we can represent the recursion with folding.

\begin{lstlisting}
isort = foldl insert []
\end{lstlisting}

Linked-list setting solution can also be described imperatively. Suppose
function \textproc{Key}($x$), returns the value of element stored in node
$x$, and \textproc{Next}($x$) accesses the next node in the linked-list.

\begin{algorithmic}
\Function{Insert}{$L, x$}
  \State $p \gets NIL$
  \State $H \gets L$
  \While{$L \neq NIL \land $ \Call{Key}{$L$} $<$ \Call{Key}{$x$}}
    \State $p \gets L$
    \State $L \gets $ \Call{Next}{$L$}
  \EndWhile
  \State \Call{Next}{$x$} $\gets L$
  \If{$p \neq NIL$}
    \State $H \gets x$
  \Else
    \State \Call{Next}{$p$} $\gets x$
  \EndIf
  \State \Return $H$
\EndFunction
\end{algorithmic}

For example in ANSI C, the linked-list can be defined as the following.

\lstset{language=C}
\begin{lstlisting}
struct node{
  Key key;
  struct node* next;
};
\end{lstlisting}

Thus the insert function can be given as below.

\begin{lstlisting}
struct node* insert(struct node* lst, struct node* x){
  struct node *p, *head;
  p = NULL;
  for(head = lst; lst && x->key > lst->key; lst = lst->next)
    p = lst;
  x->next = lst;
  if(!p)
    return x;
  p->next = x;
  return head;
}
\end{lstlisting}

Instead of using explicit linked-list such as by pointer or reference
based structure. Linked-list can also be realized by another index array.
For any array element $A_i$, $Next_i$ stores the index of next element
follows $A_i$. It means $A_{Next_i}$ is the next element after $A_i$.

The insertion algorithm based on this solution is given like below.

\begin{algorithmic}
\Function{Insert}{$A, Next, i$}
  \State $j \gets \perp$
  \While{$Next_j \neq NIL \land A_{Next_j} < A_i$}
    \State $j \gets Next_j$
  \EndWhile
  \State $Next_i \gets Next_j$
  \State $Next_j \gets i$
\EndFunction
\end{algorithmic}

Here $\perp$ means the head of the $Next$ table. 
And the relative Python program for this algorithm is given as the following.

\lstset{language=Python}
\begin{lstlisting}
def isort(xs):
    n = len(xs)
    next = [-1]*(n+1)
    for i in range(n):
        insert(xs, next, i)
    return next

def insert(xs, next, i):
    j = -1
    while next[j] != -1 and xs[next[j]] < xs[i]:
        j = next[j]
    next[j], next[i] = i, next[j]
\end{lstlisting}

Although we change the insertion operation to constant time by using 
linked-list. However, we have to traverse the linked-list to find the
position, which results $O(N^2)$ times comparison. This is because
linked-list, unlike array, doesn't support random access. It means we
can't use binary search with linked-list setting.

\begin{Exercise}
\begin{itemize}
\item Complete the insertion sort by using linked-list insertion function
in your favorate imperative programming language.
\item The index based linked-list return the sequence of rearranged index
as result. Write a program to re-order the original array of elements from
this result.
\end{itemize}
\end{Exercise}

% ================================================================
% Final improvement
% ================================================================

\section{Final improvement by binary search tree}
\index{Insertion sort!binary search tree}

It seems that we drive into a corner. We must improve both the comparison 
and the insertion at the same time, or we will end up with $O(N^2)$ performance.

We must use binary search, this is the only way to improve the comparison
time to $O(\lg N)$. On the other hand, we must change the data structure, 
because we can't achieve constant time insertion at a position with 
plain array.

This remind us about our 'hello world' data structure, binary search tree.
It naturally support binary search from its definition. At the same time,
We can insert a new leaf in binary search tree in $O(1)$ constant time
if we already find the location.

So the algorithm changes to this.

\begin{algorithmic}
\Function{Sort}{$A$}
  \State $T \gets \Phi$
  \For{each $x \in A$}
    \State $T \gets $ \Call{Insert-Tree}{$T, x$}
  \EndFor
  \State \Return \Call{To-List}{$T$}
\EndFunction
\end{algorithmic}

Where \textproc{Insert-Tree}() and \textproc{To-List}() are described in 
previous chapter about binary search tree.

As we have analyzed for binary search tree, the performance of tree sort
is bound to $O(N \lg N)$, which is the lower limit of comparison based 
sort\cite{Knuth}.

\section{Short summary} 
In this chapter, we present the evolution process of insertion sort. Insertion
sort is well explained in most textbooks as the first sorting algorithm.
It has simple and straightforward idea, but the performance is quadratic.
Some textbooks stop here, but we want to show that there exist ways to improve
it by different point of view. We first try to save the comparison time
by using binary search, and then try to save the insertion operation by
changing the data structure to linked-list. Finally, we combine these
two ideas and evolute insertion sort to tree sort.

\begin{thebibliography}{99}

\bibitem{wiki-bubble-sort}
http://en.wikipedia.org/wiki/Bubble\_sort

\bibitem{CLRS}
Thomas H. Cormen, Charles E. Leiserson, Ronald L. Rivest and Clifford Stein. 
``Introduction to Algorithms, Second Edition''. ISBN:0262032937. The MIT Press. 2001

\bibitem{Knuth}
Donald E. Knuth. ``The Art of Computer Programming, Volume 3: Sorting and Searching (2nd Edition)''. Addison-Wesley Professional; 2 edition (May 4, 1998) ISBN-10: 0201896850 ISBN-13: 978-0201896855

\end{thebibliography}

\ifx\wholebook\relax\else
\end{document}
\fi
