\documentclass{article}
\usepackage{tikz}

\usepackage{listings}
\usepackage{algorithm}
\usepackage[noend]{algpseudocode} %for pseudo code, include algorithmicsx automatically

\begin{document}

\title{Circular linked-list}
\author{Larry LIU Xinyu}
\maketitle

In imperative settings, a linked-list may be corrupted, that it is circular. In such list, some node
points back to previous one. Figure \ref{fig:circular-list} shows such situation.
The normal iteration ends up infinite looping.
  \begin{enumerate}
    \item Write a program to detect if a linked-list is circular;
    \item Write a program to find the node where the loop starts (the node being pointed by two precedents).
  \end{enumerate}

\begin{figure}[htdp]
\centering
\begin{tikzpicture}[scale=3]
  % trace
  \draw[dashed, very thin] (-2cm, 1cm) -- (0, 1cm);
  \draw[dashed, very thin] (0,0) circle [radius=1cm];

  % leading nodes
  \foreach \x in {-2, -1.7, ..., -1.4} {
    \draw[thick] (\x cm, 1cm) +(-0.1, -0.1) rectangle ++(0.1, 0.1);
    \draw[thick, ->] (\x cm, 1cm) -- +(0.2, 0);
  }

  % cricular starting points
  \draw[thick, ->] (-0.3cm, 1cm) -- (-0.1cm, 1cm);

  % circular nodes
  \foreach \deg/\rot in {90/0, 60/-30, 30/-60, 0/-90, 180/90, 150/60, 120/30} {
    \draw[thick] (\deg : 1cm) +(-0.1, -0.1) rectangle ++(0.1, 0.1);
    \draw[thick, ->] (\deg : 1cm) -- +(\rot : 0.2);
  }
\end{tikzpicture}
\caption{A circular linked-list}
\label{fig:circular-list}
\end{figure}

The brute-force solution utilizes extra spaces to record the nodes being visited so-far.
When visits a node, If it has been visited, then the linked-list is circular, and this
node is the staring point of the loop. When arrives at the tail. The linked-list isn't
circular. This method isn't good enough. The space need is $O(n)$, where $n$ is the
number of nodes in the linked-list. Depends on what data-structure is used to store
the visited node, the performance is from $O(n \lg n)$ (using map) to quadratic $O(n^2)$
(using list for example).

Consier a clock. The hand of hours rotates slower than the hand of minutes. Starts
from 12:00, they meets 12 times before the next 12:00. For this problem,
we can mimic the clock hands by using two pointers. They iterate in different speed.
If there is a loop in a linked-list, the slower pointer must be caught by the faster
one at some time. Is it true? For some circle contains $n$ nodes, the
slower pointer iterate $k$ nodes a step, the faster pointer iterate $m$ nodes a step.
Where $n$, $m$, $k$ are all natual numbers. What if $am \not\equiv bk (\textrm{mod}\ n)$ for
any integer $a, b$? We can set the slower speed as $v_s = 1$ node per step.
and $v_f = 2$ nodes per step. Starts from the head of the linked-list, if they
meet at some time, the linked-list is circular. Otherwise, if the faster pointer
arrives at or goes beyound the tail, the linked-list isn't circular.

\begin{algorithmic}[1]
\Function{Circular?}{$L$}
  \State $p \gets q \gets L$
  \While{$p \neq $ NIL $\land q \neq$ NIL}
    \State $p \gets$ \Call{Next}{$p$}
    \State $q \gets$ \Call{Next}{$q$}
    \If{$q=$ NIL}
      \State break
    \EndIf
    \State $q \gets$ \Call{Next}{$q$}
    \If{$p = q$}
      \State \Return True
    \EndIf
  \EndWhile
  \State \Return False
\EndFunction
\end{algorithmic}

The following ANSI C example program implements this method.

\lstset{language=C}
\begin{lstlisting}
int is_circular(struct Node* h) {
    struct Node *a, *b;
    a = b = h;
    while (a && b) {
        a = a->next;
        b = b->next;
        if (!b)
            break;
        b = b->next;
        if (a == b) return 1;
    }
    return 0;
}
\end{lstlisting}

\begin{figure}[htdp]
\centering
\begin{tikzpicture}[scale=3]
  % trace
  \draw[dashed, very thin] (-2cm, 1cm) -- (0, 1cm);
  \draw[dashed, very thin] (0,0) circle [radius=1cm];

  % label line
  \draw[very thin] (0, 0) -- (0, 1.4cm);
  \draw[very thin] (0, 0) -- (15:1.4cm);
  \draw[very thin] (-2cm, 1cm) -- +(0, 0.4cm);

  % label and arrows
  \draw[very thin, <->] (-2cm, 1.2cm) -- (-1cm, 1.2cm) node[above]{$k$} -- (0, 1.2cm) node[above]{$A, v_s$};
  \draw[very thin, <->] (0, 1.2cm) arc [start angle = 90, end angle = 15, radius=1.2cm] node[right]{$B, v_f = 2v_s$};
  \draw (45:1.2cm) node[above]{$k$};
  \draw[very thin, <->] (15:0.5cm) arc [start angle = 15, end angle = -270, radius=0.5cm];
  \draw (-135:0.5cm) node[left]{$n-k$};

  % leading nodes
  \foreach \x in {-2, -1.7, ..., -1.4} {
    \draw[thick] (\x cm, 1cm) +(-0.1, -0.1) rectangle ++(0.1, 0.1);
    \draw[thick, ->] (\x cm, 1cm) -- +(0.2, 0);
  }

  % cricular starting points
  \draw[thick, ->] (-0.3cm, 1cm) -- (-0.1cm, 1cm);

  % circular nodes
  \foreach \deg/\rot in {90/0, 60/-30, 30/-60, 0/-90, 180/90, 150/60, 120/30} {
    \draw[thick] (\deg : 1cm) +(-0.1, -0.1) rectangle ++(0.1, 0.1);
    \draw[thick, ->] (\deg : 1cm) -- +(\rot : 0.2);
  }
\end{tikzpicture}
\caption{Two pointers solution.}
\end{figure}

\end{document}
